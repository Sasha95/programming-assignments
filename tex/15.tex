\section{Графика и визуализация}

\subsection{Обработка изображений}

\task Написать программу, выполняющую гамма-коррекцию изображения. Для
этого значения компонент $R$, $G$ и $B$ каждого пикселя изменяются по
формуле
\[
x'=x^{\gamma},
\]
где $x'$ — новоое значение компоненты, $x$ — старое, а $\gamma$
задаётся пользователем.

Программа должна позволять открыть произвольное изображение на диске и
сохранить результат работы.

\task Написать программу, строящую множество Мандельброта. Для этого
на изображении размером $300 \times 200$ пикселей ставятся чёрные и
белые точки.

Для каждого пикселя с координатами $(X, Y)$
($0\leqslant X < 300, 0\leqslant Y < 200$) выполняется 50 итераций
преобразования
\begin{eqnarray*}
x_{n+1} &=& x_n^2 - y_n^2 + \frac{X - 200}{100},\\
y_{n+1} &=& 2 x_n y_n + \frac{Y-100}{100},
\end{eqnarray*}
где $x_1=y_1=0.$

Если выполняется неравенство $\sqrt{x_{50}^2 + y_{50}^2} < 2,$ то
пиксель окрашивается в чёрный цвет, иначе в белый.

Программа должна позволять сохранить результат работы на диск.

\task Написать программу, считывающую из текстового документа четвёрки
чисел — координаты концов отрезков, и строящую эти отрезки на изображении
размером $200\times 200$ пикселей.

Программа должна позволять открыть произвольный текстовый документ на
диске и сохранить результат работы.

\task Написать программу, выполняющую пороговую бинаризацию
изображения.

Для каждого пикселя вычисляется яркость по формуле
\[
Y = 0{,}299 R + 0{,}587 G + 0{,}114 B,
\]
где $(R, G, B)$ — компоненты пикселя (находятся в диапазоне от $0$ до
$255$).

Если $Y > 128,$ то пиксель заменяется на белый, иначе на чёрный.

Программа должна позволять открыть произвольное изображение на диске и
сохранить результат работы.

\task Написать программу, отображающую на экране три цветовых канала
изображения — красный, зелёный и синий.

Каждый канал — это отдельное изображение того же размера, у которого
все компоненты кроме одной равны нулю, а ненулевая равна
соответствующей компоненте исходного изображения.

Программа должна позволять открыть произвольное изображение на диске.

\task Написать программу, размывающую изображение.  Для этого каждая
компонента каждого пикселя заменяется на среднее арифметическое
значения компонент соседних пикселей. Соседними считаются пиксели, у
которых координата отличается не более, чем на единицу. Например, у
внутренних пикселей $8$ соседей, у пикселей на границе — $5$, на углах
— $3$.

Программа должна позволять открыть произвольное изображение на диске и
сохранить результат работы.

\task Написать программу, выделяющую границы на изображении с помощью
оператора Собеля.

Для каждого пикселя с координатами $(i,j)$ (кроме крайних) вычисляются
значения
\begin{eqnarray*}
Y_{i,j} &=& 0{,}299 R_{i,j} + 0{,}587 G_{i,j} + 0{,}114 B_{i,j},\\
V_{i,j} &=& (Y_{i+1,j-1} + 2 Y_{i+1,j} + Y_{i+1,j-1}) - (Y_{i-1,j-1} + 2 Y_{i-1,j} + Y_{i-1,j-1}),\\
H_{i,j} &=& (Y_{i-1,j+1} + 2 Y_{i,j+1} + Y_{i+1,j+1}) - (Y_{i-1,j-1} + 2 Y_{i,j-1} + Y_{i+1,j-1}),\\
G_{i,j} &=& \frac{\sqrt{V_{i,j}^2+H_{i,j}^2}}{8}.
\end{eqnarray*}
Затем значения компонент пикселя заменяются на $G_{i,j}.$

Программа должна позволять открыть произвольное изображение на диске и
сохранить результат работы.

\task Написать программу, добавляющую рамку произвольному
изображению. Толщина рамки в пикселях и цвет задаются пользователем.

Программа должна позволять открыть произвольное изображение на диске и
сохранить результат работы.

\task Написать программу, преобразующую цветное изображение в
изображение в оттенках серого. Значения компонент пикселей нового
изображения вычисляются по формуле
\[
R' = G' = B' = 0{,}299 R + 0{,}587 G + 0{,}114 B,
\]
где $(R',G',B')$ — компоненты пикселей нового изображения, а $(R,G,B)$
— старого.

Программа должна позволять открыть произвольное изображение на диске и
сохранить результат работы.

\task Написать программу, строяющую гистограмму яркостей.

Для этого для каждого пикселя вычисляется яркость по формуле
\[
Y = \lfloor 0{,}299 R + 0{,}587 G + 0{,}114 B \rfloor,
\]
где $(R,G,B)$ — компоненты пикселей изображения.

Затем строится график, на котором по оси абсцисс отложены значения
яркостей, а по оси ординат — доля пикселей с этой яркостью.

Программа должна позволять открыть произвольное изображение на диске.


% \subsection{Графики функций}

% \task (Наложенные графики)

% \task (Параметрический)

% \task (Дифур)

% \task (Траектория)

% \task

% \task

% \task

% \task

% \task

% \task

