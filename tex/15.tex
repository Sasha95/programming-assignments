\section{Графика и визуализация}

\subsection{Обработка изображений}

\task Написать программу, выполняющую гамма-коррекцию
изображения\index{Изображение,!гамма-коррекция}. Для этого значения
компонент $R$, $G$ и $B$ каждого пикселя изменяются по формуле
\[
x'=x^{\gamma},
\]
где $x'$ — новое значение компоненты, $x$ — старое, а $\gamma$
задаётся пользователем.

Вычисления выполнять с насыщением. То есть, если значение яркости
больше максимально допустимого, то яркость становится равной
максимально допустимому значению.

Программа должна позволять открыть произвольное изображение на диске и
сохранить результат работы.

\task Написать программу, строящую множество
Мандельброта.\index{Множество Мандельброта} Для этого на изображении
размером $300 \times 200$ пикселей ставятся чёрные и белые точки.

Для каждого пикселя с координатами $(x, y)$
($0\leqslant x < 300, 0\leqslant y < 200$) выполняется 50 итераций
преобразования
\begin{eqnarray*}
a_{n+1} &=& a_n^2 - b_n^2 + \frac{x - 200}{100},\\
b_{n+1} &=& 2 a_n b_n + \frac{y-100}{100},
\end{eqnarray*}
где $a_1=b_1=0.$

Если выполняется неравенство $\sqrt{a_{50}^2 + b_{50}^2} < 2,$ то
пиксель окрашивается в чёрный цвет, иначе в белый.

Программа должна позволять сохранить результат работы на диск.

\task Написать программу, считывающую из текстового документа четвёрки
чисел — координаты концов отрезков, и строящую эти отрезки на изображении
размером $200\times 200$ пикселей.

Программа должна позволять открыть произвольный текстовый документ на
диске и сохранить результат работы.

\task Написать программу, выполняющую пороговую бинаризацию
изображения.\index{Изображение,!пороговая бинаризация}

Для каждого пикселя вычисляется яркость по формуле
\[
Y = 0{,}299 R + 0{,}587 G + 0{,}114 B,
\]
где $(R, G, B)$ — компоненты пикселя (находятся в диапазоне от $0$ до
$255$).

Если $Y > 128,$ то пиксель заменяется на белый, иначе на чёрный.

Программа должна позволять открыть произвольное изображение на диске и
сохранить результат работы.

\task Написать программу, отображающую на экране три цветовых канала
изображения — красный, зелёный и синий.

Каждый канал — это отдельное изображение того же размера, у которого
все компоненты кроме одной равны нулю, а ненулевая равна
соответствующей компоненте исходного изображения.

Программа должна позволять открыть произвольное изображение на диске.

\task Написать программу, размывающую
изображение.\index{Изображение,!размытие} Для этого каждая компонента
каждого пикселя заменяется на среднее арифметическое значения
компонент соседних пикселей. Соседними считаются пиксели, у которых
координата отличается не более, чем на единицу. Например, у внутренних
пикселей $8$ соседей, у пикселей на границе — $5$, на углах — $3$
соседа.

Программа должна позволять открыть произвольное изображение на диске и
сохранить результат работы.

\task Написать программу, выделяющую границы на изображении с помощью
оператора Собеля.\index{Оператор Собеля}

Для каждого пикселя с координатами $(i,j)$ (кроме крайних) вычисляются
значения
\begin{eqnarray*}
Y_{i,j} &=& 0{,}299 R_{i,j} + 0{,}587 G_{i,j} + 0{,}114 B_{i,j},\\
V_{i,j} &=& (Y_{i+1,j-1} + 2 Y_{i+1,j} + Y_{i+1,j-1}) - (Y_{i-1,j-1} + 2 Y_{i-1,j} + Y_{i-1,j-1}),\\
H_{i,j} &=& (Y_{i-1,j+1} + 2 Y_{i,j+1} + Y_{i+1,j+1}) - (Y_{i-1,j-1} + 2 Y_{i,j-1} + Y_{i+1,j-1}),\\
G_{i,j} &=& \frac{\sqrt{V_{i,j}^2+H_{i,j}^2}}{8}.
\end{eqnarray*}
Затем значения компонент пикселя заменяются на $G_{i,j}.$

Программа должна позволять открыть произвольное изображение на диске и
сохранить результат работы.

\task Написать программу, добавляющую рамку произвольному
изображению. Толщина рамки в пикселях и цвет задаются пользователем.

Программа должна позволять открыть произвольное изображение на диске и
сохранить результат работы.

\task Написать программу, преобразующую цветное изображение в
изображение в оттенках серого.\index{Изображение,!преобразование в
  серое} Значения компонент пикселей нового изображения
вычисляются по формуле
\[
R' = G' = B' = 0{,}299 R + 0{,}587 G + 0{,}114 B,
\]
где $(R',G',B')$ — компоненты пикселей нового изображения, а $(R,G,B)$
— старого.

Программа должна позволять открыть произвольное изображение на диске и
сохранить результат работы.

\task Написать программу, строящую ломаную на изображении размером
$400 \times 400$ пикселей на основе строки. Начальная точка ломаной —
$(300, 200),$ начальное направление — вверх.

Строка считывается посимвольно. Символ «$F$» — рисование отрезка в
текущем направлении длиной $d=2$ пикселя, «$+$» — поворот направления на
90° по часовой стрелке, «$-$» — поворот направления на 90° против
часовой стрелке, остальные символы игнорируются.

Строка, по которой строится ломаная, образуется по следующим правилам
(такие правила называют системами Линдемайера или L-системами).

Изначально строка равна «$FX$». Затем в строке выполняются замены:
\[
\begin{aligned}
  X &\to X+YF\\
  Y &\to FX-Y
\end{aligned}
\]

Например, после первой замены строка «$FX$» преобразуется в строку
\[
FX+YF,
\]
а после второй в строку
\[
FX+YF+FX-YF.
\]

Замены повторяются $k = 15$ раз. Получившаяся кривая называется
драконом Хартера — Хейтуэя.\index{Дракон Хартера — Хейтуэя}

Программа должна позволять сохранить результат работы на диск.


\subsection{Графики функций}

\task Написать программу, строящую гистограмму яркостей пикселей
изображения.\index{Изображение!гистограмма}

Для этого для каждого пикселя вычисляется яркость по формуле
\[
Y = \lfloor 0{,}299 R + 0{,}587 G + 0{,}114 B \rfloor,
\]
где $(R,G,B)$ — компоненты пикселей изображения.

Затем строится график, на котором по оси абсцисс отложены значения
яркостей, а по оси ординат — доля пикселей с этой яркостью.

Программа должна позволять открыть произвольное изображение на диске.

\task Написать программу, строящую гипотрохоиду\index{Гипотрохоида} —
кривую, задаваемую параметрическими уравнениями:
\[
\left\{
  \begin{aligned}
    x &= \left( R - r \right) \cos t + h \cos \left( \frac{R-r}r t \right),\\
    y &= \left( R - r \right) \sin t - h \sin \left( \frac{R-r}r t \right),
  \end{aligned}
\right.
\]
где $R, r, h$ — константы, задаваемые пользователем, $t$ — параметр.

\task Написать программу, строящую график многочлена
\[
P(x) = a_nx^n+a_{n-1}x^{n-1}+\dots+a_1x+a_0,
\]
где коэффициенты $a_i$ и диапазон значений аргумента $x$ задаются
пользователем. Количество коэффициентов может быть различным.

\task Написать программу, строящую траекторию движения тела, брошенную
под углом к горизонту с учётом сопротивления воздуха.\index{Траектория
  снаряда!с сопротивлением воздуха} Траектория задаётся
параметрическими уравнениями
\[
\left\{
\begin{aligned}
x &= v_0 \cos\varphi \frac{m}{k} \left( 1 - \exp\Bigl(-\frac{k}{m}t\Bigr) \right),\\
y &= \frac{m}{k} \Biggl(
  \biggl(v_0 \sin\varphi + \frac{mg}{k}\biggr) 
  \biggl(1 - \exp\Bigl(-\frac{k}{m}t\Bigr)\biggr) 
  - gt 
\Biggr),
\end{aligned}
\right.
\]
где $v_0$ — начальная скорость, $\varphi$ — начальный угол, $m$ —
масса тела, $k$ — коэффициент сопротивления воздуха,
$g\approx 9{,}81~\frac{\textup{м}}{\textup{с}^2}$ — ускорение свободного
падения, $t$ — время.

Все параметры должны задаваться пользователем.

\task Дан текстовый файл, содержащий целые числа (по одному на
строку). Написать программу, строящую столбчатую диаграмму с частотами
первых значащих цифр этих чисел.

Например, для чисел $12, 7, 395, 117$ первые значащие цифры —
$1, 7, 3, 1$. А частоты, соответственно, —
$P[1] = 0{,}5; P[3] = 0{,}25; P[7] = 0{,}25$.

Программа может использоваться для проверки так называемого закона
Бенфорда.\index{Закон Бенфорда} Согласно ему в числах, взятых из
реальной жизни, меньшие цифры встречаются чаще в начале числа, чем
большие.

\task Метод середин квадратов,\index{Метод!фон Неймана} предложенный
фон Нейманом для генерации последовательности псевдослучайных чисел
заключается в следующем. Берётся четырёхзначное число, возводится в
квадрат и в качестве нового числа используются средние четыре цифры —
с третьей по седьмую справа. Затем действия повторяются. (В настоящее
время этот метод не используется из-за плохих статистических
характеристик получаемой последовательности.)

Например, пусть дано число $1234$. Его квадрат равен
$1\underline{5227}56$, следовательно новое число в последовательности
— $5227$.

Написать программу, строящую круговую диаграмму, показывающую
соотношение чётных и нечётных чисел в первых $N$ элементах
последовательности. Параметр $N$ и начальное значение должны
задаваться пользователем.

\task Дан текстовый файл, содержащий пары вещественных чисел,
разделённых пробелом (по одной паре на строку). Написать программу,
строящую график в декартовых координатах на основе файла. Первое число
— абсцисса, второе — ордината.

Программа должна позволять пользователю указать открываемый файл.

\task Дан текстовый файл, содержащий пары вещественных чисел,
разделённых пробелом (по одной паре на строку). Написать программу,
строящую график в полярных координатах на основе файла. Первое число —
угловая координата, второе — радиальная.

Программа должна позволять пользователю указать открываемый файл.

\task Так называемое логистическое
преобразование\index{Преобразование!логистическое} имеет вид
\[
x_{n+1} = rx_n(1-x_n).
\]
Определим $F(x_0, r)$ как значение $x_{1000}$ для заданных $x_0$ и $r$.

Пусть значение $x_0$ меняется в полуотрезке $(0;1]$, а $r$ — от $2{,}5$
до $4$ с шагом $0{,}05$.

Написать программу, строящую график, на котором для каждого значения
$x_0$ и $r$ из указанных диапазонов ставится точка с координатами
$(r, F(x_0, r))$. Полученный график называется бифуркационной
диаграммой логистического преобразования.

\task Одна из моделей, используемых для анализа численности популяции
в экологии, — модель Ферхюльста.\index{Модель Ферхюльста} Если $x_k$ —
численность в $k$-м году, то численность на следующий год можно
приближённо найти по формуле:
\[
x_{n+1} = x_n + r x_n \left( 1 - \frac{x_n}K \right),
\]
где $r$ — удельная скорость роста, а $K$ — ёмкость экологической ниши
популяции.

Написать программу, строящую график численности популяции. Необходимые
параметры должны задаваться пользователем.

