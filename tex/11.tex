\section{Графические интерфейсы}

\subsection{Однооконные приложения}

\task Написать программу с графическим пользовательским интерфейсом
для решения линейных и квадратных уравнений. Пользователь должен иметь
возможность выбрать тип уравнения и ввести коэффициенты.

В интерфейсе должна быть предусмотрена проверка корректности входных
данных. Компоненты, которые не требуются для решения выбранной задачи
должны быть неактивны или невидимы.

\task Написать программу с графическим пользовательским интерфейсом,
которая определяет, насколько один цвет в формате RGB близок к
другому. Цвета запрашивать с помощью стандартного диалога. (Значения
компонентов — целые числа из диапазона от 0 до 255.)

Для проверки близости цветов $(R_1, G_1, B_1)$ и $(R_2, G_2, B_2)$
использовать формулу
\[
\rho = {\sqrt{(R_1-R_2)^2 + (G_1-G_2)^2 + (B_1-B_2)^2} \over 255 \cdot \sqrt{3}}.
\]

В графическом пользовательском интерфейсе должна быть предусмотрена
проверка корректности входных данных.

\task Написать программу с графическим пользовательским интерфейсом,
которая для введённого текста находит частоту использования каждой
буквы. Регистр букв при расчёте не учитывать.

Список букв с частотами должен быть упорядочен по убыванию частоты.

\task Написать программу с графическим пользовательским интерфейсом
для решения треугольников. То есть, программа должна находить стороны
и углы треугольника
\begin{itemize}
\item по трём сторонам,
\item по двум сторонам и углу между ними,
\item по стороне и двум прилежащим углам.
\end{itemize}

В графическом пользовательском интерфейсе должна быть предусмотрена
проверка корректности входных данных. Пользователь должен иметь
возможность выбрать тип решаемой задачи. Компоненты, которые не
требуются для решения выбранной задачи должны быть неактивны или
невидимы.

\task Написать программу с графическим пользовательским интерфейсом,
которая вычисляет сопротивление провода $R$ по длине $l$ и площади
поперечного сечения $S$. Расчёты выполняются по формуле
\[
R = \frac{\rho l}{S},
\]
где $\rho$ — удельное сопротивление материала (значения для часто
используемых материалов приведены в таблице \ref{table:resistance}).

\begin{table}
  \centering
  \begin{tabular}{l|l}
    Материал&$\rho, \frac{\textrm{Ом}\cdot\textrm{мм}^2}{\textrm{м}}$\\
    \hline
    Серебро  & 0{,}015 \\
    Медь     & 0{,}018 \\
    Золото   & 0{,}023 \\
    Алюминий & 0{,}028 \\
    Сталь    & 0{,}120 \\
    Нихром   & 1{,}200
  \end{tabular}
  \caption{Удельные сопротивления для различных материалов}
  \label{table:resistance}
\end{table}

Пользователь должен иметь возможность выбрать материал провода из списка.

В графическом пользовательском интерфейсе должна быть предусмотрена
проверка корректности входных данных.

\task Написать программу с графическим пользовательским интерфейсом
для расчёта суммы денежного вклада на 12~месяцев.

Входные данные — первоначальный взнос и процентная
ставка. Пользователь должен иметь возможность выбрать, начисляются ли
проценты на первоначальную или текущую сумму на счету («сложные
проценты»).

В графическом пользовательском интерфейсе должна быть предусмотрена
проверка корректности входных данных.

\task Написать программу с графическим пользовательским интерфейсом,
которая применяла бы указанное регулярное выражение к введённому
тексту.

Пользователь должен иметь возможность выбрать, проверяется ли
соответствие текста регулярному выражению или должна выполняться
замена. Во втором случае должен указываться текст, на который
заменяются найденные фрагменты.

В графическом пользовательском интерфейсе компоненты, которые не
требуются для решения выбранной задачи должны быть неактивны или
невидимы.

\task Написать программу с графическим пользовательским интерфейсом
для перевода температуры между кельвинами, градусами Цельсия и
Фаренгейта.

Температура в кельвинах $t_K$ и градусах Фаренгейта $t_F$ выражается
через температуру в градусах Цельсия $t_C$ следующим образом:
\begin{align*}
  t_F &= \frac95 t_C + 32,\\
  t_K &= t_C + 273{,}15.
\end{align*}

В графическом пользовательском интерфейсе должна быть предусмотрена
проверка корректности входных данных. Пользователь должен иметь
возможность выбрать направление перевода.

\task Написать программу с графическим пользовательским интерфейсом
для расчёта площади треугольника, круга, трапеции или прямоугольника
по длинам сторон. Пользователь должен иметь возможность выбрать
геометрическую фигуру.

В графическом пользовательском интерфейсе должна быть предусмотрена
проверка корректности входных данных. Компоненты, которые не требуются
для решения выбранной задачи должны быть неактивны или невидимы.

\task Написать программу с графическим пользовательским интерфейсом,
исправляющую текст, набранный в неправильной раскладке. (Например,
русский текст, набранный при включённой английской раскладке.)

