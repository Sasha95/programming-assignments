\section{Графические интерфейсы}

\subsection{Однооконные приложения}

\task Написать программу с графическим пользовательским интерфейсом
для решения линейных и квадратных уравнений. Пользователь должен иметь
возможность выбрать тип уравнения и ввести коэффициенты.

В интерфейсе должна быть предусмотрена проверка корректности входных
данных. Компоненты, которые не требуются для решения выбранной задачи
должны быть неактивны или невидимы.

\task Написать программу с графическим пользовательским интерфейсом,
которая определяет, насколько один цвет в формате RGB близок к
другому. (Значения компонентов — целые числа из диапазона от 0 до 255.)

Для проверки близости цветов $(R_1, G_1, B_1)$ и $(R_2, G_2, B_2)$
использовать формулу
\[
\rho = {\sqrt{(R_1-R_2)^2 + (G_1-G_2)^2 + (B_1-B_2)^2} \over 255 \cdot \sqrt{3}}.
\]

В графическом пользовательском интерфейсе должна быть предусмотрена
проверка корректности входных данных.

\task Написать программу с графическим пользовательским интерфейсом,
которая для введённого текста находит частоту использования каждой
буквы. Регистр букв при расчёте не учитывать.

Список букв с частотами должен быть упорядочен по убыванию частоты.

\task Написать программу с графическим пользовательским интерфейсом
для решения треугольников. То есть, программа должна находить стороны
и углы треугольника
\begin{itemize}
\item по трём сторонам,
\item по двум сторонам и углу между ними,
\item по стороне и двум прилежащим углам.
\end{itemize}

В графическом пользовательском интерфейсе должна быть предусмотрена
проверка корректности входных данных. Пользователь должен иметь
возможность выбрать тип решаемой задачи. Компоненты, которые не
требуются для решения выбранной задачи должны быть неактивны или
невидимы.

\task Написать программу с графическим пользовательским интерфейсом,
которая вычисляет сопротивление провода $R$ по длине $l$ и площади
поперечного сечения $S$. Расчёты выполняются по формуле
\[
R = \frac{\rho l}{S},
\]
где $\rho$ — удельное сопротивление материала (значения для часто
используемых материалов приведены в таблице \ref{table:resistance}).

\begin{table}
  \centering
  \begin{tabular}{l|l}
    Материал&$\rho, \frac{\textrm{Ом}\cdot\textrm{мм}^2}{\textrm{м}}$\\
    \hline
    Серебро  & 0{,}015 \\
    Медь     & 0{,}018 \\
    Золото   & 0{,}023 \\
    Алюминий & 0{,}028 \\
    Сталь    & 0{,}120 \\
    Нихром   & 1{,}200
  \end{tabular}
  \caption{Удельные сопротивления для различных материалов}
  \label{table:resistance}
\end{table}

Пользователь должен иметь возможность выбрать материал провода из списка.

В графическом пользовательском интерфейсе должна быть предусмотрена
проверка корректности входных данных.

\task Написать программу с графическим пользовательским интерфейсом
для расчёта суммы денежного вклада на 12~месяцев.

Входные данные — первоначальный взнос и процентная
ставка. Пользователь должен иметь возможность выбрать, начисляются ли
проценты на первоначальную или текущую сумму на счету («сложные
проценты»).

В графическом пользовательском интерфейсе должна быть предусмотрена
проверка корректности входных данных.

\task Написать программу с графическим пользовательским интерфейсом,
которая применяла бы указанное регулярное выражение к введённому
тексту.

Пользователь должен иметь возможность выбрать, проверяется ли
соответствие текста регулярному выражению или должна выполняться
замена. Во втором случае должен указываться текст, на который
заменяются найденные фрагменты.

В графическом пользовательском интерфейсе компоненты, которые не
требуются для решения выбранной задачи должны быть неактивны или
невидимы.

\task Написать программу с графическим пользовательским интерфейсом
для перевода температуры между кельвинами, градусами Цельсия и
Фаренгейта.

Температура в кельвинах $t_K$ и градусах Фаренгейта $t_F$ выражается
через температуру в градусах Цельсия $t_C$ следующим образом:
\begin{align*}
  t_F &= \frac95 t_C + 32,\\
  t_K &= t_C + 273{,}15.
\end{align*}

В графическом пользовательском интерфейсе должна быть предусмотрена
проверка корректности входных данных. Пользователь должен иметь
возможность выбрать направление перевода.

\task Написать программу с графическим пользовательским интерфейсом
для расчёта площади треугольника, круга, трапеции или прямоугольника
по длинам сторон. Пользователь должен иметь возможность выбрать
геометрическую фигуру.

В графическом пользовательском интерфейсе должна быть предусмотрена
проверка корректности входных данных. Компоненты, которые не требуются
для решения выбранной задачи должны быть неактивны или невидимы.

\task Написать программу с графическим пользовательским интерфейсом,
исправляющую текст, набранный в неправильной раскладке. (Например,
русский текст, набранный при включённой английской раскладке.)


\subsection{Диалоги и взаимодействие форм}

\task Написать программу с графическим пользовательским интерфейсом,
которая находит сумму площадей прямоугольников, введённых
пользователем. Для ввода размеров прямоугольника предусмотреть
отдельную форму.

\task Написать программу с графическим пользовательским интерфейсом,
выводящую сведения о введённом пользователем тексте: количество
символов включая и исключая пробелы. Сведения должны отображаться в
отдельном окне.

\task Написать программу с графическим пользовательским интерфейсом,
которая для вводимых пользователем значений роста и веса группы людей
находит минимальное и максимальное значения индекса массы тела.

Индекс массы тела вычисляется по формуле
\[
I=\frac{m}{h^2},
\]
где $m$ — масса (кг), $h$ — рост (м).

Для ввода роста и веса предусмотреть отдельную форму.

\task Написать программу с графическим пользовательским интерфейсом
для поиска общего сопротивления цепи из параллельных или
последовательно соединённых резисторов.

Для ввода сопротивления предусмотреть отдельную форму. Пользователь
должен иметь возможность выбрать тип соединения резисторов.

\task Написать программу с графическим пользовательским интерфейсом,
выводящую на экран списка задач на день с указанием приоритета
(«важная», «обычная», «не срочно»).

Для ввода текста задачи и приоритета использовать отдельную
форму. Пользователь должен выбирать приоритет из списка.

\task Написать программу с графическим пользовательским интерфейсом,
которая для указанного цвета выводит значения его компонент в формате
$\mathrm{RGB}$ и в формате $\mathrm{Y'C_BC_R}$.

Пусть компоненты $\mathrm{RGB}$ — однобайтовые целые, тогда компоненты
$\mathrm{Y'C_BC_R}$ вычисляются по формулам
\begin{align*}
  Y'  &=  16 +65{,}481 \cdot R + 128{,}553 \cdot G + 24{,}966 \cdot B\\
  C_B &= 128 -37{,}797 \cdot R - 74{,}203  \cdot G + 112{,}0  \cdot B\\
  C_R &= 128 +112{,}0  \cdot R - 93{,}786  \cdot G - 18{,}214 \cdot B.
\end{align*}

Цвет запрашивать с помощью стандартного диалога. Выбранный цвет должен
отображаться на главной форме в виде закрашенного квадрата.

\task Написать программу с графическим пользовательским интерфейсом
для вычисления значения многочлена в указанной точке. Каждый
коэффициент при степени и сама степень переменной вводятся с помощью
отдельной формы.

\task Написать программу с графическим пользовательским интерфейсом
для определения длины ломаной на плоскости. Точки, составляющие
ломаную, последовательно вводятся пользователем.

Для ввода координат точек предусмотреть отдельную форму. 

\task Написать программу с графическим пользовательским интерфейсом
для вычисления координат центра масс точек на плоскости.

Для ввода координат и массы точек предусмотреть отдельную форму.

\task Написать программу с графическим пользовательским интерфейсом,
которая для указанного цвета подбирает цвет, дополняющий его до
белого. Цвет запрашивать с помощью стандартного диалога. Оба цвета
должны отображаться на главной форме в виде закрашенных квадратов.


