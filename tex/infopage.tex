\newlength{\UDCLength}
\setlength{\UDCLength}{\maxof{\widthof{УДК}}{\widthof{ББК}}}
\noindent\makebox[\UDCLength][l]{УДК} \UDC\\
\noindent\makebox[\UDCLength][l]{ББК} \BBK\\
\noindent\makebox[\UDCLength][l]{} \AS

\vfill{}

\noindent
    {\addfontfeature{LetterSpace=30}Рецензенты}:

\noindent\textbf{Калинкова~Е.~В.}, ст. преп. каф. ПМиИ ПГУ им. Т.~Г.~Шевченко

\noindent\textbf{Марков~Д.~А.}, к.ф.-м.н., нач. УИР ПГУ им.~Т.~Г.~Шевченко

\vfill{}

\newlength{\ASShiftLength}
\setlength{\ASShiftLength}{-0.5em-\widthof{\AS}}
\noindent {\hspace{\ASShiftLength}\AS}\hfill%
\begin{minipage}[t]{1\columnwidth}%
  \noindent\hspace{2em}\AuthorI\ \Title: \PubType\ —
  Тирасполь,~\Year. —
  \begin{NoHyper}{\pageref{LastPage}}\end{NoHyper}~с.
  \medskip{}

  \hspace{2em}{\small Сборник содержит задачи для вводного курса
    программирования для направлений подготовки, связанных с
    разработкой программного обеспечения. Задачи охватывают широкий
    круг тем и парадигм: структурное, объектно-ориентированное,
    функциональное программирование. Хотя задачи не привязаны к
    конкретному языку программирования, предполагается, что они будут
    решаться на языке C\#.}

  \hspace{2em}{\small Пособие предназначено для студентов, обучающихся
    по направлениям «Прикладная математика и информатика», «Прикладная
    математика» и других.}
\end{minipage}

\noindent
\begin{flushright}
  \begin{minipage}[t]{0.5\columnwidth}
    \noindent\makebox[\UDCLength][l]{УДК} \UDC\\
    \noindent\makebox[\UDCLength][l]{ББК} \BBK
  \end{minipage}
\end{flushright}

\vfill{}

\begin{center}
  Утверждено Научно-методическим советом ПГУ~им.~Т.~Г.~Шевченко
\end{center}

\vfill{}

\noindent {\small © \AuthorI, \Year.}

\thispagestyle{empty}
\newpage
