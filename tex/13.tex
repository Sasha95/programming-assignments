\section{Рекурсивные структуры данных}

\subsection{Списки}

\task Написать класс, реализующий односвязный список. Класс должен
содержать методы \Lst{Push} (добавление элемента в стек), \Lst{Pop}
(извлечение элемента из стека). Также он должен переопределять
унаследованный метод \Lst{ToString}, возвращающий строковое
представление состояния стека.

Написать метод \Lst{Sum}, вычисляющим сумму элементов списка, которые
удовлетворяют заданному критерию. Критерий задаётся функцией,
передаваемой как аргумент.

Написать программу, использующую класс.

\task Написать класс, реализующий односвязный список. Класс должен
содержать методы \Lst{Push} (добавление элемента в стек), \Lst{Pop}
(извлечение элемента из стека). Также он должен переопределять
унаследованный метод \Lst{ToString}, возвращающий строковое
представление состояния стека.

Написать метод \Lst{Last}, извлекающий последний элемент списка.

Написать программу, использующую класс.

\task Написать класс, реализующий односвязный список. Класс должен
содержать методы \Lst{Push} (добавление элемента в стек), \Lst{Pop}
(извлечение элемента из стека). Также он должен переопределять
унаследованный метод \Lst{ToString}, возвращающий строковое
представление состояния стека.

Написать метод \Lst{Reverse}, обращающий порядок следования элементов
списка.

Написать программу, использующую класс.

\task Написать класс, реализующий односвязный список. Класс должен
содержать методы \Lst{Push} (добавление элемента в стек), \Lst{Pop}
(извлечение элемента из стека). Также он должен переопределять
унаследованный метод \Lst{ToString}, возвращающий строковое
представление состояния стека.

Написать метод \Lst{Swap}, меняющий местами первый и последний
элементы списка.

Написать программу, использующую класс.

\task Написать класс, реализующий односвязный список. Класс должен
содержать методы \Lst{Push} (добавление элемента в стек), \Lst{Pop}
(извлечение элемента из стека). Также он должен переопределять
унаследованный метод \Lst{ToString}, возвращающий строковое
представление состояния стека.

Написать метод \Lst{Remove}, удаляющий элементы списка, которые
удовлетворяют заданному критерию. Критерий задаётся функцией,
передаваемой как аргумент. 

Написать программу, использующую класс.

\task Написать класс, реализующий односвязный список. Класс должен
содержать методы \Lst{Push} (добавление элемента в стек), \Lst{Pop}
(извлечение элемента из стека). Также он должен переопределять
унаследованный метод \Lst{ToString}, возвращающий строковое
представление состояния стека.

Написать метод \Lst{Append}, добавляющий указанное значение в конец
списка.

Написать программу, использующую класс.

\task Написать класс, реализующий односвязный список. Класс должен
содержать методы \Lst{Push} (добавление элемента в стек), \Lst{Pop}
(извлечение элемента из стека). Также он должен переопределять
унаследованный метод \Lst{ToString}, возвращающий строковое
представление состояния стека.

Написать метод \Lst{Extend}, добавляющий к списку копию другого
списка.

Написать программу, использующую класс.

\task Написать класс, реализующий односвязный список. Класс должен
содержать методы \Lst{Push} (добавление элемента в стек), \Lst{Pop}
(извлечение элемента из стека). Также он должен переопределять
унаследованный метод \Lst{ToString}, возвращающий строковое
представление состояния стека.

Написать метод \Lst{Count}, возвращающим количество элементов в
списке, которые удовлетворяют заданному критерию. Критерий задаётся
функцией, передаваемой как аргумент.

Написать программу, использующую класс.

\task Написать класс, реализующий односвязный список. Класс должен
содержать методы \Lst{Push} (добавление элемента в стек), \Lst{Pop}
(извлечение элемента из стека). Также он должен переопределять
унаследованный метод \Lst{ToString}, возвращающий строковое
представление состояния стека.

Написать метод \Lst{InsertAt}, добавляющий элемент в указанною позицию
списка.

Написать программу, использующую класс.

\task Написать класс, реализующий односвязный список. Класс должен
содержать методы \Lst{Push} (добавление элемента в стек), \Lst{Pop}
(извлечение элемента из стека). Также он должен переопределять
унаследованный метод \Lst{ToString}, возвращающий строковое
представление состояния стека.

Написать метод \Lst{RemoveAt}, удаляющий элемент списка с указанным
номером.

Написать программу, использующую класс.


\subsection{Деревья}

\task 

\task

\task

\task

\task

\task

\task

\task

\task

\task

