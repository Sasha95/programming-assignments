\section{Рекурсивные структуры данных}

\subsection{Списки}

\task Написать класс \Lst{SinglyLinkedList}, реализующий односвязный
список. Класс должен содержать методы \Lst{Push} (добавление элемента
в начало списка), \Lst{Pop} (извлечение элемента из начала
списка). Также он должен переопределять унаследованный метод
\Lst{ToString}, возвращающий строковое представление списка.

Написать метод \Lst{Sum}, вычисляющий сумму элементов списка, которые
удовлетворяют заданному критерию. Критерий задаётся функцией,
передаваемой как аргумент.

Написать программу, использующую класс.

\task Написать класс \Lst{SinglyLinkedList}, реализующий односвязный
список. Класс должен содержать методы \Lst{Push} (добавление элемента
в начало списка), \Lst{Pop} (извлечение элемента из начала
списка). Также он должен переопределять унаследованный метод
\Lst{ToString}, возвращающий строковое представление списка.

Написать метод \Lst{Last}, извлекающий последний элемент списка.

Написать программу, использующую класс.

\task Написать класс \Lst{SinglyLinkedList}, реализующий односвязный
список. Класс должен содержать методы \Lst{Push} (добавление элемента
в начало списка), \Lst{Pop} (извлечение элемента из начала
списка). Также он должен переопределять унаследованный метод
\Lst{ToString}, возвращающий строковое представление списка.

Написать метод \Lst{Reverse}, обращающий порядок следования элементов
списка.

Написать программу, использующую класс.

\task Написать класс \Lst{SinglyLinkedList}, реализующий односвязный
список. Класс должен содержать методы \Lst{Push} (добавление элемента
в начало списка), \Lst{Pop} (извлечение элемента из начала
списка). Также он должен переопределять унаследованный метод
\Lst{ToString}, возвращающий строковое представление списка.

Написать метод \Lst{Swap}, меняющий местами первый и последний
элементы списка.

Написать программу, использующую класс.

\task Написать класс \Lst{SinglyLinkedList}, реализующий односвязный
список. Класс должен содержать методы \Lst{Push} (добавление элемента
в начало списка), \Lst{Pop} (извлечение элемента из начала
списка). Также он должен переопределять унаследованный метод
\Lst{ToString}, возвращающий строковое представление списка.

Написать метод \Lst{Remove}, удаляющий элементы списка, которые
удовлетворяют заданному критерию. Критерий задаётся функцией,
передаваемой как аргумент. 

Написать программу, использующую класс.

\task Написать класс \Lst{SinglyLinkedList}, реализующий односвязный
список. Класс должен содержать методы \Lst{Push} (добавление элемента
в начало списка), \Lst{Pop} (извлечение элемента из начала
списка). Также он должен переопределять унаследованный метод
\Lst{ToString}, возвращающий строковое представление списка.

Написать метод \Lst{Append}, добавляющий указанное значение в конец
списка.

Написать программу, использующую класс.

\task Написать класс \Lst{SinglyLinkedList}, реализующий односвязный
список. Класс должен содержать методы \Lst{Push} (добавление элемента
в начало списка), \Lst{Pop} (извлечение элемента из начала
списка). Также он должен переопределять унаследованный метод
\Lst{ToString}, возвращающий строковое представление списка.

Написать метод \Lst{Extend}, добавляющий к списку копию другого
списка.

Написать программу, использующую класс.

\task Написать класс \Lst{SinglyLinkedList}, реализующий односвязный
список. Класс должен содержать методы \Lst{Push} (добавление элемента
в начало списка), \Lst{Pop} (извлечение элемента из начала
списка). Также он должен переопределять унаследованный метод
\Lst{ToString}, возвращающий строковое представление списка.

Написать метод \Lst{Count}, возвращающим количество элементов в
списке, которые удовлетворяют заданному критерию. Критерий задаётся
функцией, передаваемой как аргумент.

Написать программу, использующую класс.

\task Написать класс \Lst{SinglyLinkedList}, реализующий односвязный
список. Класс должен содержать методы \Lst{Push} (добавление элемента
в начало списка), \Lst{Pop} (извлечение элемента из начала
списка). Также он должен переопределять унаследованный метод
\Lst{ToString}, возвращающий строковое представление списка.

Написать метод \Lst{InsertAt}, добавляющий элемент в указанною позицию
списка.

Написать программу, использующую класс.

\task Написать класс \Lst{SinglyLinkedList}, реализующий односвязный
список. Класс должен содержать методы \Lst{Push} (добавление элемента
в начало списка), \Lst{Pop} (извлечение элемента из начала
списка). Также он должен переопределять унаследованный метод
\Lst{ToString}, возвращающий строковое представление списка.

Написать метод \Lst{RemoveAt}, удаляющий элемент списка с указанным
номером.

Написать программу, использующую класс.


\subsection{Деревья}

\task Написать класс \Lst{SearchTree}, реализующий бинарное дерево
поиска. Класс должен содержать методы \Lst{Add} (добавление элемента в
дерево), \Lst{InOrderWalk} (симметричный обход дерева).

Написать метод \Lst{LeafCount}, находящий количество листьев дерева.

Написать программу, использующую этот класс.

\task Написать класс \Lst{SearchTree}, реализующий бинарное дерево
поиска. Класс должен содержать методы \Lst{Add} (добавление элемента в
дерево), \Lst{InOrderWalk} (симметричный обход дерева).

Написать метод \Lst{Search}, проверяющий, есть ли в дереве указанный
элемент, и возвращающий путь к нему от корня. Путь представляет собой
строку, состоящую из букв «L» (поворот налево) и «R» (поворот
направо).

Написать программу, использующую этот класс.

\task Написать класс \Lst{SearchTree}, реализующий бинарное дерево
поиска. Класс должен содержать методы \Lst{Add} (добавление элемента в
дерево), \Lst{InOrderWalk} (симметричный обход дерева).

Написать метод \Lst{IsIdealBalanced}, проверяющий, является ли дерево
идеально сбалансированным.

Написать программу, использующую этот класс.

\task Написать класс \Lst{SearchTree}, реализующий бинарное дерево
поиска. Класс должен содержать методы \Lst{Add} (добавление элемента в
дерево), \Lst{InOrderWalk} (симметричный обход дерева).

Написать метод \Lst{RightRotate}, выполняющий правый поворот дерева
относительно корня. Процедура правого поворота показана на рисунке
\ref{fig:rotate}.

\begin{figure}
  \begin{centering}
    \tikzstyle{line} = [draw, -latex']
    \usetikzlibrary{shapes,positioning}
    \begin{tikzpicture}[
      thick,
      node distance=1.5cm,
      text height=1.2ex,
      text depth=.1ex,
      auto]

      \node[circle, draw]                    (B)  {B};
      \node[below right of=B]                (c)  {$\gamma$};
      \node[circle, draw, below left of=B]   (A)  {A};
      \node[below left  of=A]                (a)  {$\alpha$};
      \node[below right of=A]                (b)  {$\beta$};
      
      \path[line] (A) edge (a);
      \path[line] (A) edge (b);
      \path[line] (B) edge (A);
      \path[line] (B) edge (c);

      \node at (2, -1)          {$\longrightarrow$};

      \node[circle, draw] at (4,0)           (rA) {A};
      \node[below left of=rA]                (ra) {$\alpha$};
      \node[circle, draw, below right of=rA] (rB) {B};
      \node[below left  of=rB]               (rb) {$\beta$};
      \node[below right of=rB]               (rc) {$\gamma$};
      
      \path[line] (rA) edge (ra);
      \path[line] (rA) edge (rB);
      \path[line] (rB) edge (rb);
      \path[line] (rB) edge (rc);
    \end{tikzpicture}
    \par
  \end{centering}

  \caption{Правый поворот дерева (греческими буквами обозначены
    поддеревья)\label{fig:rotate}}
\end{figure}

Написать программу, использующую этот класс.

\task Написать класс \Lst{SearchTree}, реализующий бинарное дерево
поиска. Класс должен содержать методы \Lst{Add} (добавление элемента в
дерево), \Lst{InOrderWalk} (симметричный обход дерева).

Написать метод \Lst{Print}, выводящий строковое представление дерева в
текстовый поток (например для последующей записи в файл). Поток должен
передаваться через аргументы метода.

Пример строкового представления дерева приведён на рисунке
\ref{fig:tree-sample}.

\begin{figure}
  \centering
  \begin{minipage}[c]{0.4\textwidth}
    \centering
    \tikzstyle{line} = [draw, -latex']
    \usetikzlibrary{shapes,positioning}
    \begin{tikzpicture}[
      thick,
      node distance=1.2cm,
      text height=1.2ex,
      text depth=.1ex,
      auto]
      
      \node                    (A) {2};
      \node[below left of=A]   (C) {1};
      \node[below right of=A]  (B) {4};
      \node[below left  of=B]  (D) {3};
      \node[below right of=B]  (E) {5};
      
      \path[line] (A) edge (C);
      \path[line] (A) edge (B);
      \path[line] (B) edge (D);
      \path[line] (B) edge (E);
    \end{tikzpicture}
  \end{minipage}%
  \begin{minipage}[c]{0.2\textwidth}
    \begin{framed}
      \centering
\begin{verbatim}
      5
   4
      3
2
   1
\end{verbatim}
    \end{framed}
  \end{minipage}
  
  
  \caption{Пример строкового представления дерева
    сортировки\label{fig:tree-sample}}
\end{figure}

\begin{centering}
\end{centering}

Написать программу, использующую этот класс.

\task Написать класс \Lst{SearchTree}, реализующий бинарное дерево
поиска. Класс должен содержать методы \Lst{Add} (добавление элемента в
дерево), \Lst{InOrderWalk} (симметричный обход дерева).

Написать метод \Lst{Trim}, удаляющий все элементы, уровень которых
ниже указанного. Уровень (целое число) передаётся через аргументы
метода.

Написать программу, использующую этот класс.

\task Написать класс \Lst{SearchTree}, реализующий бинарное дерево
поиска. Класс должен содержать методы \Lst{Add} (добавление элемента в
дерево), \Lst{InOrderWalk} (симметричный обход дерева).

Написать метод \Lst{Flip}, изменяющий значение каждого узла на
противоположное и зеркально «отражающий» дерево. То есть, все
левые подузлы должнны стать правыми и наоборот.

Написать программу, использующую этот класс.

\task Написать класс \Lst{SearchTree}, реализующий бинарное дерево
поиска. Класс должен содержать методы \Lst{Add} (добавление элемента в
дерево), \Lst{InOrderWalk} (симметричный обход дерева).

Написать метод \Lst{GetByPath}, возвращающий элемент, находящийся в
дереве по указанному пути. Путь представляет собой
строку, состоящую из букв «L» (поворот налево) и «R» (поворот
направо).

Написать программу, использующую этот класс.

\task Написать класс \Lst{SearchTree}, реализующий бинарное дерево
поиска. Класс должен содержать методы \Lst{Add} (добавление элемента в
дерево), \Lst{InOrderWalk} (симметричный обход дерева).

Написать метод \Lst{Height}, возвращающий высоту дерева, то есть
максимальную длину от корня до листа.

Написать программу, использующую этот класс.

\task Написать класс \Lst{SearchTree}, реализующий бинарное дерево
поиска. Класс должен содержать методы \Lst{Add} (добавление элемента в
дерево), \Lst{InOrderWalk} (симметричный обход дерева).

Написать метод \Lst{Sum}, вычисляющим сумму элементов дерева, которые
удовлетворяют заданному критерию. Критерий задаётся функцией,
передаваемой как аргумент.

Написать программу, использующую этот класс.