\section{Рекурсивные структуры данных}

\subsection{Списки}

\task Написать класс, реализующий односвязный список. Класс должен
содержать методы \Lst{Push} (добавление элемента в начало списка),
\Lst{Pop} (извлечение элемента из начала списка). Также он должен
переопределять унаследованный метод \Lst{ToString}, возвращающий
строковое представление списка.

Написать метод \Lst{Sum}, вычисляющим сумму элементов списка, которые
удовлетворяют заданному критерию. Критерий задаётся функцией,
передаваемой как аргумент.

Написать программу, использующую класс.

\task Написать класс, реализующий односвязный список. Класс должен
содержать методы \Lst{Push} (добавление элемента в начало списка),
\Lst{Pop} (извлечение элемента из начала списка). Также он должен
переопределять унаследованный метод \Lst{ToString}, возвращающий
строковое представление списка.

Написать метод \Lst{Last}, извлекающий последний элемент списка.

Написать программу, использующую класс.

\task Написать класс, реализующий односвязный список. Класс должен
содержать методы \Lst{Push} (добавление элемента в начало списка),
\Lst{Pop} (извлечение элемента из начала списка). Также он должен
переопределять унаследованный метод \Lst{ToString}, возвращающий
строковое представление списка.

Написать метод \Lst{Reverse}, обращающий порядок следования элементов
списка.

Написать программу, использующую класс.

\task Написать класс, реализующий односвязный список. Класс должен
содержать методы \Lst{Push} (добавление элемента в начало списка),
\Lst{Pop} (извлечение элемента из начала списка). Также он должен
переопределять унаследованный метод \Lst{ToString}, возвращающий
строковое представление списка.

Написать метод \Lst{Swap}, меняющий местами первый и последний
элементы списка.

Написать программу, использующую класс.

\task Написать класс, реализующий односвязный список. Класс должен
содержать методы \Lst{Push} (добавление элемента в начало списка),
\Lst{Pop} (извлечение элемента из начала списка). Также он должен
переопределять унаследованный метод \Lst{ToString}, возвращающий
строковое представление списка.

Написать метод \Lst{Remove}, удаляющий элементы списка, которые
удовлетворяют заданному критерию. Критерий задаётся функцией,
передаваемой как аргумент. 

Написать программу, использующую класс.

\task Написать класс, реализующий односвязный список. Класс должен
содержать методы \Lst{Push} (добавление элемента в начало списка),
\Lst{Pop} (извлечение элемента из начала списка). Также он должен
переопределять унаследованный метод \Lst{ToString}, возвращающий
строковое представление списка.

Написать метод \Lst{Append}, добавляющий указанное значение в конец
списка.

Написать программу, использующую класс.

\task Написать класс, реализующий односвязный список. Класс должен
содержать методы \Lst{Push} (добавление элемента в начало списка),
\Lst{Pop} (извлечение элемента из начала списка). Также он должен
переопределять унаследованный метод \Lst{ToString}, возвращающий
строковое представление списка.

Написать метод \Lst{Extend}, добавляющий к списку копию другого
списка.

Написать программу, использующую класс.

\task Написать класс, реализующий односвязный список. Класс должен
содержать методы \Lst{Push} (добавление элемента в начало списка),
\Lst{Pop} (извлечение элемента из начала списка). Также он должен
переопределять унаследованный метод \Lst{ToString}, возвращающий
строковое представление списка.

Написать метод \Lst{Count}, возвращающим количество элементов в
списке, которые удовлетворяют заданному критерию. Критерий задаётся
функцией, передаваемой как аргумент.

Написать программу, использующую класс.

\task Написать класс, реализующий односвязный список. Класс должен
содержать методы \Lst{Push} (добавление элемента в начало списка),
\Lst{Pop} (извлечение элемента из начала списка). Также он должен
переопределять унаследованный метод \Lst{ToString}, возвращающий
строковое представление списка.

Написать метод \Lst{InsertAt}, добавляющий элемент в указанною позицию
списка.

Написать программу, использующую класс.

\task Написать класс, реализующий односвязный список. Класс должен
содержать методы \Lst{Push} (добавление элемента в начало списка),
\Lst{Pop} (извлечение элемента из начала списка). Также он должен
переопределять унаследованный метод \Lst{ToString}, возвращающий
строковое представление списка.

Написать метод \Lst{RemoveAt}, удаляющий элемент списка с указанным
номером.

Написать программу, использующую класс.


\subsection{Деревья}

\task Написать класс, реализующий бинарное дерево поиска. Класс должен
содержать методы \Lst{Add} (добавление элемента в дерево),
\Lst{InOrderTraverse} (симметричный обход дерева).

Написать метод \Lst{LeafCount}, находящий количество листьев дерева.

Написать программу, использующую этот класс.

\task Написать класс, реализующий бинарное дерево поиска. Класс должен
содержать методы \Lst{Add} (добавление элемента в дерево),
\Lst{InOrderTraverse} (симметричный обход дерева).

Написать метод \Lst{Search}, проверяющий, есть ли в дереве указанный
элемент, и возвращающий путь к нему от корня. Путь представляет собой
строку, состоящую из букв «L» (поворот налево) и «R» (поворот
направо).

Написать программу, использующую этот класс.

\task Написать класс, реализующий бинарное дерево поиска. Класс должен
содержать методы \Lst{Add} (добавление элемента в дерево),
\Lst{InOrderTraverse} (симметричный обход дерева).

Написать метод \Lst{IsIdealBalanced}, проверяющий, является ли дерево идеально сбалансированным.

Написать программу, использующую этот класс.

\task Написать класс, реализующий бинарное дерево поиска. Класс должен
содержать методы \Lst{Add} (добавление элемента в дерево),
\Lst{InOrderTraverse} (симметричный обход дерева).

Написать метод \Lst{FromArray}, формирующий дерево на основе массива
по слоям. То есть, первый элемент массива — корень, второй и третий —
его потомки, следующие четыре — потомки второго и третьего и т.~д.

Написать программу, использующую этот класс.

\task Написать класс, реализующий бинарное дерево поиска. Класс должен
содержать методы \Lst{Add} (добавление элемента в дерево),
\Lst{InOrderTraverse} (симметричный обход дерева).

Написать метод \Lst{Print}, выводящий дерево в текстовый поток. Поток
должен передаваться через аргументы метода.

Написать программу, использующую этот класс.

\task Написать класс, реализующий бинарное дерево поиска. Класс должен
содержать методы \Lst{Add} (добавление элемента в дерево),
\Lst{InOrderTraverse} (симметричный обход дерева).

Написать метод \Lst{Trim}, удаляющий все элементы, уровень которых
ниже указанного. Уровень (целое число) передаётся через аргументы
метода.

Написать программу, использующую этот класс.

\task Написать класс, реализующий бинарное дерево поиска. Класс должен
содержать методы \Lst{Add} (добавление элемента в дерево),
\Lst{InOrderTraverse} (симметричный обход дерева).

Написать метод, изменяющий значение каждого узла на результат
применения к нему некоторой функции. Функция передаётся через
аргументы.

Написать программу, использующую этот класс.

\task Написать класс, реализующий бинарное дерево поиска. Класс должен
содержать методы \Lst{Add} (добавление элемента в дерево),
\Lst{InOrderTraverse} (симметричный обход дерева).

Написать метод \Lst{GetByPath}, возвращающий элемент, находящийся в
дереве по указанному пути. Путь представляет собой
строку, состоящую из букв «L» (поворот налево) и «R» (поворот
направо).

Написать программу, использующую этот класс.

\task Написать класс, реализующий бинарное дерево поиска. Класс должен
содержать методы \Lst{Add} (добавление элемента в дерево),
\Lst{InOrderTraverse} (симметричный обход дерева).

Написать метод \Lst{Height}, возвращающий высоту дерева, то есть
максимальную длину от корня до листа.

Написать программу, использующую этот класс.

\task Написать класс, реализующий бинарное дерево поиска. Класс должен
содержать методы \Lst{Add} (добавление элемента в дерево),
\Lst{InOrderTraverse} (симметричный обход дерева).

Написать метод \Lst{Sum}, вычисляющим сумму элементов дерева, которые
удовлетворяют заданному критерию. Критерий задаётся функцией,
передаваемой как аргумент.

Написать программу, использующую этот класс.