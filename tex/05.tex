\section{Классы и исключения}

\subsection{Класс}

\subsection{Перегрузка операций}

\task Описать класс комплексных чисел Complex. Комплексные числа имеют
вид $a+bi,$ где $i=\sqrt{-1},$ $a,b \in \mathbb{R}$. Определить в нем:
\begin{itemize*}
\item конструктор, принимающий действительную и мнимую часть;
\item копирующий конструктор;
\item методы Re и Im, возвращающие мнимую и действительную части;
\item методы Abs и Arg, возвращающие модуль и аргумент числа;
\item операции сложения, вычитания, умножения и деления (аргументы
  могут быть как комплексными, так и комплексным и действительным
  числами);
\item перегруженный метод ToString.
\end{itemize*}

Написать программу, использующую этот класс.

\task Описать класс интервалов времени TimeInterval. Определить в нем:
\begin{itemize*}
\item конструктор, принимающий часы, минуты и секунды (должен
  корректно обрабатывать случаи, когда количество секунд или минут
  больше 60, или когда они отрицательные);
\item копирующий конструктор;
\item метод IsWorkTime, проверяющий, относится ли момент, отстоящий на
  указанный интервал от полуночи, к рабочему времени (между 8:00 и
  17:00).
\item метод IsGreater, проверяющий, является ли интервал более
  продолжительным по сравнению с другим;
\item операции сложения и вычитания;
\item операции умножения и деления на число;
\item перегруженный метод ToString.
\end{itemize*}

Написать программу, использующую этот класс.

\task Описать класс Matrix2 матриц вида
$\begin{pmatrix}
a_{11} & a_{12} \\
a_{21} & a_{22}
\end{pmatrix},$ где $a_{ij} \in \mathbb{R}$. Определить в нем:
\begin{itemize*}
\item конструктор, принимающий четыре элемента матрицы;
\item конструктор, принимающий два элемента главной диагонали
  (остальные элементы равны нулю);
\item копирующий конструктор;
\item метод Det, возвращающий определитель матрицы;
\item метод Inverse, возвращающий обратную матрицу;
\item метод Transpose, возвращающий транспонированную матрицу;
\item операции сложения и вычитания матриц;
\item операции умножения и деления (аргументы могут быть как матрицами,
  так и матрицей и действительным числом);
\item перегруженный метод ToString.
\end{itemize*}

Написать программу, использующую этот класс.

\task Описать класс Polynomial3 многочленов вида $ax^2+bx+c,$ где
$a,b,c \in \mathbb{R}$. Определить в нем:
\begin{itemize*}
\item конструктор, принимающий коэффициенты многочлена;
\item копирующий конструктор;
\item метод RealZerosNumber, возвращающий количество нулей многочлена
  (ноль, один, два или бесконечно много в случае $a=b=c=0$);
\item операции сложения и вычитания;
\item операции умножения и деления на действительное число;
\item операцию вычисления остатка от деления одного многочлена на
  другой;
\item перегруженный метод ToString.
\end{itemize*}

Написать программу, использующую этот класс.

\task Описать класс Vector2 векторов на плоскости. Определить в нем:
\begin{itemize*}
\item конструктор, принимающий координаты вектора;
\item копирующий конструктор;
\item метод Length, возвращающий длину вектора;
\item метод Angle, вычисляющий угол между текущим и другим вектором;
\item операции сложения и вычитания;
\item операцию скалярного умножения вектора на вектор;
\item операции умножения и деления на целое число;
\item перегруженный метод ToString.
\end{itemize*}

Написать программу, использующую этот класс.

\task Описать класс Money денежных сумм, заданных в виде количества рублей и копеек.
 Определить в нем:
\begin{itemize*}
\item конструктор, принимающий количество рублей и копеек (должен
  корректно обрабатывать случаи, когда копеек больше 100, или
  количества рублей и копеек имеют разные знаки);
\item копирующий конструктор;
\item метод TransferCost, принимающий величину комиссии за денежный
  перевод в процентах и возвращающий его полную стоимость с точностью
  до копеек (например, для суммы 10~р.~15~к. и величины комиссии 5~\%
  полная стоимость составляет 10~р.~66~к.);
\item операции сложения и вычитания;
\item операции умножения и деления на действительное число (результат
  должен округляться до копеек);
\item перегруженный метод ToString.
\end{itemize*}

Написать программу, использующую этот класс.

\task Описать класс Fraction дробей вида $\frac{m}{n},$ где $m \in
\mathbb{Z}, n \in \mathbb{N}$.  Определить в нем:
\begin{itemize*}
\item конструктор, принимающий числитель и знаменатель дроби (должен
  приводить дробь к несократимому виду);
\item копирующий конструктор;
\item метод IntegerPart, возвращающий целую часть дроби;
\item операции сложения, вычитания, умножения и деления (аргументы
  могут быть как дробями, так и дробью и целым числом);
\item перегруженный метод ToString.
\end{itemize*}

Написать программу, использующую этот класс.

\task В европейской музыке весь диапазон музыкальных звуков делится на
октавы, которые можно пронумеровать числами от $-3$ до $5$. Октавы с
неположительными номерами имеют собственные названия:
субконтроктава~($-3$), контроктава~($-2$), большая октава~($-1$),
малая октава~($0$).

В каждой октаве 12~музыкальных звуков (нот): до, до-диез, ре, ре-диез,
ми, фа, фа-диез, соль, соль-диез, ля, ля-диез, си.

Описать класс Note музыкальных звуков. Определить в нем:
\begin{itemize*}
\item конструктор, принимающий номер октавы и номер звука (считать,
  что нота до имеет номер $0$);
\item копирующий конструктор;
\item метод Frequency, возвращающий частоту ноты (нота ля первой
  октавы имеет частоту 440~Гц, частота каждой следующей ноты больше в
  $\sqrt[12]{2}$~раз);
\item операции прибавления и вычитания целого числа, позволяющие
  получить следующие и предыдущие звуки в общей последовательности;
\item перегруженный метод ToString, возвращающий текстовое описание
  звука (например, «до-диез 2-й октавы» или «ля субконтроктавы»).
\end{itemize*}

Написать программу, использующую этот класс.

\task Описать класс дат Date. Определить следующие методы:
\begin{itemize*}
\item конструктор, принимающий номера дня, месяца и года
  (рассматривать только положительные номера года);
\item копирующий конструктор;
\item методы Next и Prev, возвращающие следующий или предыдущий день;
\item метод IsLeapYear, проверяющий, является ли текущий год
  високосным;
\item операции прибавления и вычитания целого числа, позволяющие
  получить следующие за текущим или предыдущие дни;
\item перегруженный метод ToString, возвращающий текстовое
  представление даты (например, «15 апреля 1707 г.»).
\end{itemize*}

Написать программу, использующую этот класс.

\task Дуальным числом называется число вида $a+b\varepsilon,$ где $a,
b \in \mathbb{R},$ а $\varepsilon \neq 0$ — абстрактная величина,
такая что $\varepsilon^2 = 0$.

Описать класс DualNumber дуальных чисел. Определить в нем:
\begin{itemize*}
\item конструктор, принимающий компоненты $a$ и $b$ дуального числа;
\item копирующий конструктор;
\item метод Pow, позволяющий возвести дуальное число в произвольную
  натуральную степень;
\item операции сложения, вычитания, умножения и деления (аргументы
  могут быть как дуальными числами, так и дуальным и действительным
  числами);
\item перегруженный метод ToString.
\end{itemize*}

Написать программу, использующую этот класс.

\subsection{Исключения}
