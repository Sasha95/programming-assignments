\section{Классы и исключения}

\subsection{Структуры}

\task Опишите структуру Sequence, соответствующую числовым
последовательностям. Последовательность задаётся начальным членом
$a_0$ и функцией $f$. Каждый член кроме первого определяется через
предыдущий:
\[
a_{k+1} = f(a_k).
\]
Напишите функцию, находящую $n$-й член последовательности.

Напишите программу, использующую эту функцию. Предусмотрите обработку
исключительных ситуаций.

\task Опишите структуру Point3, соответствующую точкам в
пространстве. Определите функцию, проверяющую, лежат ли три точки на
одной прямой.

Напишите программу, использующую эту функцию. Предусмотрите обработку
исключительных ситуаций.

\task Опишите структуру Interval, соответствующую отрезкам числовой
прямой. Отрезок $[a, b]$ задаётся своими концами. Определите функцию,
возвращающую длину общей части отрезков или ноль, если они не
пересекаются.

Напишите программу, использующую эту функцию. Предусмотрите обработку
исключительных.

\task Опишите структуру Circle, соответствующую
окружностям. Окружность задаётся координатами центра и
радиусом. Определите функцию, проверяющую, пересекаются ли две
окружности.

Напишите программу, использующую эту функцию. Предусмотрите обработку
исключительных ситуаций.

\task Опишите структуру GeometricProgression, соответствующую
геометрическим прогрессиям. Определите функцию, находящую сумму
бесконечного числа членов прогрессии.

Напишите программу, использующую эту функцию. Предусмотрите обработку
исключительных ситуаций.

\task Опишите структуру, соответствующую датам григорианского
календаря после 1583~г. Дата задаётся тройкой: день $d$, месяц $m$ и
год $y.$ Определите функцию, возвращающую название дня недели,
соответствующего дате.

Номер дня недели $N$ можно вычислить по следующим формулам.
\begin{eqnarray*}
a &=& \frac{14 - m}{12},\\
Y &=& y - a,\\
M &=& m + 12a - 2,\\
N &\equiv& 7000 + d + Y + \frac{Y}{4} - \frac{Y}{100} + \frac{Y}{400} + \frac{31}{12}M \mod 7.
\end{eqnarray*}
Если $N$ равен 0, то результат — воскресенье, 1 — понедельник, 2 —
вторник и т.~д.

Напишите программу, использующую эту функцию. Предусмотрите обработку
исключительных ситуаций.

\task Опишите структуру Triangle, соответствующую
треугольникам. Треугольник задаётся длинами сторон $a$, $b$ и $c$.
Определите функцию, вычисляющую его углы.

Напишите программу, использующую эту функцию. Предусмотрите обработку
исключительных ситуаций.

\task Опишите структуру Point2, соответствующую точкам на
плоскости. Определите функцию, вычисляющую расстояние от некоторой
точки до прямой, заданной двумя другими точками.

Напишите программу, использующую эту функцию. Предусмотрите обработку
исключительных ситуаций.

\task Опишите структуру Time, соответствующую моментам времени. Время
задаётся как тройка: часы ($h$), минуты ($m$) и секунды
($s$). Определите функцию, вычисляющую количество секунд между двумя
моментами.

Напишите программу, использующую эту функцию. Предусмотрите обработку
исключительных ситуаций.

\task Опишите структуру Line, соответствующую прямым на
плоскости. Прямая задаётся коэффициентами $A$, $B$ и $С$ уравнения
\[
Ax+By+C=0.
\]
Определите функцию, находящую координаты точки пересечения двух
прямых.

Напишите программу, использующую эту функцию. Предусмотрите обработку
исключительных ситуаций.


\subsection{Исключения}

\task Опишите функцию умножения двух целых, обработайте ошибку переполнения сверху (overflow).

\task Опишите функцию деления двух целых, обработайте ошибку переполнения снизу (underflow).

\task Опишите функцию деления двух целых, обработайте ошибку деления на ноль (zero division).

\task Переопределите оператор ++ для указателя на массив целых, обработайте ошибку выхода за границы массива.

\task Опишите функцию анализа номера телефона, обработайте ошибку задания номера в неверном формате (допустимый формат - +7(095)555-44-33).

\task 

\task Опишите функцию, возвращающую день недели по дню и месяцу, обработайте ошибки неверного дня или месяца.

\task Опишите функцию умножения двух чисел с плавающей запятой, обработайте ошибку переполнения сверху (overflow).

\task Опишите функцию деления двух чисел с плавающей запятой, обработайте ошибку переполнения снизу (underflow).

\task Опишите функцию деления двух чисел с плавающей запятой, обработайте ошибку деления на ноль (zero division).


\subsection{Классы и перегрузка операций}

\task Описать класс комплексных чисел Complex. Комплексные числа имеют
вид $a+bi,$ где $i=\sqrt{-1},$ $a,b \in \mathbb{R}$. Определить в нем:
\begin{itemize*}
\item конструктор, принимающий действительную и мнимую часть;
\item копирующий конструктор;
\item методы Re и Im, возвращающие мнимую и действительную части;
\item методы Abs и Arg, возвращающие модуль и аргумент числа;
\item операции сложения, вычитания, умножения и деления (аргументы
  могут быть как комплексными, так и комплексным и действительным
  числами);
\item перегруженный метод ToString.
\end{itemize*}

Предусмотреть возможные исключительные ситуации, если это необходимо.

Написать программу, использующую этот класс.

\task Описать класс отрезков числовой прямой Interval. Определить в нем:
\begin{itemize*}
\item конструктор, принимающий концы отрезка (должен корректно
  обрабатывать случаи, когда левый конец больше правого);
\item копирующий конструктор;
\item метод Length, возвращающий длину отрезка.
\item операции интервальной арифметики;
\item перегруженный метод ToString.
\end{itemize*}

Операции интервальной арифметики определяются следующим образом:
\begin{align*}
  [a, b] + [c, d] &= [a + c, b + d],\\
  [a, b] - [c, d] &= [a - c, b - d],\\
  [a, b] \times [c, d] &=
  [\min \{ac, ad, bc, bd\}, \max \{ac, ad, bc, bd\}],\\
  \frac{[a, b]}{[c, d]} &=
  \left[
    \min \left\{\frac{a}{c}, \frac{a}{d}, \frac{b}{c}, \frac{b}{d}\right\},
    \max \left\{\frac{a}{c}, \frac{a}{d}, \frac{b}{c}, \frac{b}{d}\right\}
    \right], \textrm{если } 0\not\in[c, d].
\end{align*}

Предусмотреть возможные исключительные ситуации, если это необходимо.

Написать программу, использующую этот класс.

\task Описать класс Matrix2 матриц вида
$\begin{pmatrix}
a_{11} & a_{12} \\
a_{21} & a_{22}
\end{pmatrix},$ где $a_{ij} \in \mathbb{R}$. Определить в нем:
\begin{itemize*}
\item конструктор, принимающий четыре элемента матрицы;
\item конструктор, принимающий два элемента главной диагонали
  (остальные элементы равны нулю);
\item копирующий конструктор;
\item метод Det, возвращающий определитель матрицы;
\item метод Inverse, возвращающий обратную матрицу;
\item метод Transpose, возвращающий транспонированную матрицу;
\item операции сложения и вычитания матриц;
\item операции умножения и деления (аргументы могут быть как матрицами,
  так и матрицей и действительным числом);
\item перегруженный метод ToString.
\end{itemize*}

Предусмотреть возможные исключительные ситуации, если это необходимо.

Написать программу, использующую этот класс.

\task Описать класс Polynomial2 квадратных многочленов вида
$ax^2+bx+c,$ где $a,b,c \in \mathbb{R}$. Определить в нем:
\begin{itemize*}
\item конструктор, принимающий коэффициенты многочлена;
\item копирующий конструктор;
\item метод Value, возвращающий значение многочлена в заданной точке;
\item операции сложения и вычитания;
\item операции умножения и деления на действительное число;
\item операцию вычисления остатка от деления одного многочлена на
  другой;
\item перегруженный метод ToString.
\end{itemize*}

Предусмотреть возможные исключительные ситуации, если это необходимо.

Написать программу, использующую этот класс.

\task Описать класс Vector3 векторов в пространстве. Определить в нем:
\begin{itemize*}
\item конструктор, принимающий координаты вектора;
\item копирующий конструктор;
\item метод Length, возвращающий длину вектора;
\item метод Angle, вычисляющий угол между текущим и другим вектором;
\item операции сложения и вычитания;
\item операцию скалярного умножения вектора на вектор;
\item операции умножения и деления на целое число;
\item перегруженный метод ToString.
\end{itemize*}

Предусмотреть возможные исключительные ситуации, если это необходимо.

Написать программу, использующую этот класс.

\task Описать класс Money денежных сумм, заданных в виде количества рублей и копеек.
 Определить в нем:
\begin{itemize*}
\item конструктор, принимающий количество рублей и копеек (должен
  корректно обрабатывать случаи, когда копеек больше 100, или
  количества рублей и копеек имеют разные знаки);
\item копирующий конструктор;
\item метод TransferCost, принимающий величину комиссии за денежный
  перевод в процентах и возвращающий его полную стоимость с точностью
  до копеек (например, для суммы 10~р.~15~к. и величины комиссии 5~\%
  полная стоимость составляет 10~р.~66~к.);
\item операции сложения и вычитания;
\item операции умножения и деления на действительное число (результат
  должен округляться до копеек);
\item перегруженный метод ToString.
\end{itemize*}

Предусмотреть возможные исключительные ситуации, если это необходимо.

Написать программу, использующую этот класс.

\task Описать класс Fraction дробей вида $\frac{m}{n},$ где $m \in
\mathbb{Z}, n \in \mathbb{N}$.  Определить в нем:
\begin{itemize*}
\item конструктор, принимающий числитель и знаменатель дроби (должен
  приводить дробь к несократимому виду);
\item копирующий конструктор;
\item метод IntegerPart, возвращающий целую часть дроби;
\item операции сложения, вычитания, умножения и деления (аргументы
  могут быть как дробями, так и дробью и целым числом);
\item перегруженный метод ToString.
\end{itemize*}

Предусмотреть возможные исключительные ситуации, если это необходимо.

Написать программу, использующую этот класс.

\task В европейской музыке весь диапазон музыкальных звуков делится на
октавы, которые можно пронумеровать числами от $-3$ до $5$. Октавы с
неположительными номерами имеют собственные названия:
субконтроктава~($-3$), контроктава~($-2$), большая октава~($-1$),
малая октава~($0$).

В каждой октаве 12~музыкальных звуков (нот): до, до-диез, ре, ре-диез,
ми, фа, фа-диез, соль, соль-диез, ля, ля-диез, си.

Описать класс Note музыкальных звуков. Определить в нем:
\begin{itemize*}
\item конструктор, принимающий номер октавы и номер звука (считать,
  что нота до имеет номер $0$);
\item копирующий конструктор;
\item метод Frequency, возвращающий частоту ноты (нота ля первой
  октавы имеет частоту 440~Гц, частота каждой следующей ноты больше в
  $\sqrt[12]{2}$~раз);
\item операции прибавления и вычитания целого числа, позволяющие
  получить следующие и предыдущие звуки в общей последовательности;
\item перегруженный метод ToString, возвращающий текстовое описание
  звука (например, «до-диез 2-й октавы» или «ля субконтроктавы»).
\end{itemize*}

Предусмотреть возможные исключительные ситуации, если это необходимо.

Написать программу, использующую этот класс.

\task Описать класс дат Date. Определить следующие методы:
\begin{itemize*}
\item конструктор, принимающий номера дня, месяца и года
  (рассматривать только положительные номера года);
\item копирующий конструктор;
\item методы Next и Prev, возвращающие следующий или предыдущий день;
\item метод IsLeapYear, проверяющий, является ли текущий год
  високосным;
\item операции прибавления и вычитания целого числа, позволяющие
  получить следующие за текущим или предыдущие дни;
\item перегруженный метод ToString, возвращающий текстовое
  представление даты (например, «15 апреля 1707 г.»).
\end{itemize*}

Предусмотреть возможные исключительные ситуации, если это необходимо.

Написать программу, использующую этот класс.

\task Дуальным числом называется число вида $a+b\varepsilon,$ где $a,
b \in \mathbb{R},$ а $\varepsilon \neq 0$ — абстрактная величина,
такая что $\varepsilon^2 = 0$.

Описать класс DualNumber дуальных чисел. Определить в нем:
\begin{itemize*}
\item конструктор, принимающий компоненты $a$ и $b$ дуального числа;
\item копирующий конструктор;
\item метод Pow, позволяющий возвести дуальное число в произвольную
  натуральную степень;
\item операции сложения, вычитания, умножения и деления (аргументы
  могут быть как дуальными числами, так и дуальным и действительным
  числами);
\item перегруженный метод ToString.
\end{itemize*}

Операции над дуальными числами определены следующим образом:
\begin{align*}
  (a+b\varepsilon) + (c+d\varepsilon) &= (a+c) + (b+d)\varepsilon,\\
  (a+b\varepsilon) - (c+d\varepsilon) &= (a-c) + (b-d)\varepsilon,\\
  (a+b\varepsilon) \times (c+d\varepsilon) &= (ac) + (bc+ad)\varepsilon,\\
  \frac{a+b\varepsilon}{c+d\varepsilon} &=
  \frac{a}{c} + \frac{bc-ad}{c^2}\varepsilon.
\end{align*}

Предусмотреть возможные исключительные ситуации, если это необходимо.

Написать программу, использующую этот класс.
