\section{Циклы}

\subsection{Цикл с параметром}

\task На счёте в банке лежит $S~\textup{руб}$. Через каждый месяц
размер вклада увеличивается на $p\,\%.$ Написать программу, выводящую
таблицу с суммой вклада в каждом из следующих 12~месяцев.

\task Последовательность чисел $a_0, a_1, a_2, \dots$ образуется по
закону:
\begin{align*}
a_0 &= 1,\\
a_k &= ka_{k-1} + \frac1k.
\end{align*}
Написать программу, вычисляющую $a_n$ для заданного номера $n.$

\task Начав тренировки спортсмен в первый день пробежал
$L~\textup{км}.$ Каждый следующий день он увеличивал пробег на $p\,\%$
от пробега предыдущего дня. Написать программу, определяющую суммарный
пробег за $N$~дней.

\task Последовательность Фибоначчи определяется следующим образом:
\begin{align*}
a_0 &= 1,\\
a_1 &= 1,\\
a_k &= a_{k-1} + a_{k-2}.
\end{align*}
Написать программу, выводящую первые $n$ членов последовательности.

\task Дано натуральное число $n$. Написать программу, вычисляющую
сумму
\[
n^2+(n+1)^2+\ldots+(2n)^2.
\]

\task Написать программу, вычисляющую частичную сумму гармонического
ряда
\[
1+\frac{1}{2}+\frac{1}{3}+\ldots+\frac1n
\]
для заданного $n.$

\task Написать программу, вычисляющую сумму
\[
\frac12 + \frac23 + \ldots + \frac{n}{n+1}
\]
для заданного $n.$

\task Известно сопротивление каждого из элементов электрической
цепи. Все элементы соединены параллельно. Написать программу,
вычисляющую общее сопротивление цепи.

\task Двойной факториал числа определяется следующим образом:
\[
n!! =
\begin{cases}
  1 \cdot 3 \cdot 5 \cdot \ldots \cdot n, &\textup{если }n\textup{ нечётное},\\
  2 \cdot 4 \cdot 6 \cdot \ldots \cdot n, &\textup{если }n\textup{ чётное}.
\end{cases}
\]
Написать программу, вычисляющую двойной факториал для заданного
натурального $n$.

\task Написать программу, вычисляющую для заданного $n$ значение суммы
\[
1+\frac1{1!} + \frac{1}{2!} + \ldots + \frac1{n!}.
\]

\subsection{Цикл с условием}

\task Написать программу, находящую наибольший общий делитель двух
чисел с помощью алгоритма Евклида.

\task Дано действительное число $a.$ Написать программу, находящую
такое наименьшее число $n,$ что
\[
1 + \frac12 + \frac13 + \ldots + \frac1n > a.
\]

\task Дано натуральное число. Написать программу, находящую
арифметическое среднее его цифр.

\task Приближённое решение уравнения $\cos x = x$ можно найти как
предел последовательности, заданной рекуррентным соотношением:
\begin{align*}
  a_0 &= 1,\\
  a_k &= \cos a_{k-1}.
\end{align*}
Дано малое действительное число $\varepsilon > 0$. Написать программу,
находящую первый член последовательности, для которого $|a_k - \cos
a_k| < \varepsilon.$

\task Написать программу, проверяющую, является ли введённое значение
факториалом некоторого числа.

\task Дана последовательность чисел $a_1, a_2, \ldots, a_n.$ Написать
программу, проверяющую, является ли последовательность упорядоченной
по возрастанию. Если это не так, то вывести номер первого числа,
нарушающего упорядоченность.

\task Написать программу, находящую наименьший делитель заданного
числа, отличный от $1.$

\task Написать программу, находящую сумму заданного натурального числа
$n$ и числа, полученного записью цифр числа $n$ в обратном порядке.

\task Дано натуральное число $n$ и цифра $k.$ Написать программу,
находящую номер первого вхождения цифры в число. (Позиции
отсчитываются справа налево начиная с 0.)

\task Дано натуральное число. Написать программу, проверяющую является
ли она палидромом. (Число называется палидромом, если оно записывается
одинаково слева направо и справа налево.)

\subsection{Проверка условия внутри цикла}

\task Написать программу, находящую наибольшую и наименьшую цифры
заданного натурального числа.

\task Число называется автоморфным, если оно равно последним цифрам
своего квадрата. Например, $5^2=25$. Написать программу, находящую все
трёхзначные автоморфные числа.

\task Написать программу, находящую все двузначные числа, сумма
квадратов цифр которых кратна 13.

\task Написать программу, определяющую сколько раз цифра $k$ входит в
десятичную запись заданного числа $n$.

\task Число называется совершенным, если оно равно сумме своих
делителей за исключением его самого. Например, $6=1+2+3=1\cdot
2\cdot3$. Написать программу, проверяющую, является ли указанное число
совершенным.

\task Написать программу, удаляющую из записи числа все чётные
цифры. Например, из числа $12345$ должно получиться $135$.

\task Написать программу, проверяющую для заданного числа, упорядочены
ли его цифры по возрастанию справа налево.

\task Написать программу, находящую количество делителей заданного
числа $n$.

\task Дано натуральное число $n$ и действительные числа $a_1, a_2,
\ldots, a_n$. Написать программу, вычисляющую
\[
\max_{1\leqslant i\leqslant n} \left|a_i\right|.
\]

\task Написать программу, находящую все двузначные числа, равные
утроенному произведению своих цифр.

\subsection{Вложенные циклы}

\task Написать программу, находящую первые 100 простых чисел.

\task Некоторая последовательность начинается с натурального числа,
кратного 3. Каждый следующий член равен сумме кубов цифр
предыдущего. Известно, что любая такая последовательность начиная с
некоторого члена становится постоянной, и её члены равны
$153$. Написать программу, находящую количество членов, не равных $153$
для указанного первого члена.

\task Написать программу, находящую все натуральные корни уравнения
\[
a^2+b^2=c^2,
\]
где $a, b, c \in \left\{1, 2, 3, \ldots, 10\right\}.$

\task Дано натуральное число $n$. Написать программу, выводящую его
разложение на простые множители.

\task Даны натуральные числа $m$ и $n$. Написать программу, находящую
все натуральные числа, меньшие $n,$ квадрат суммы цифр которых равен
$m$.

\task Натуральное $n$-значное число называется числом Армстронга, если
сумма его цифр, возведённых в $n$-ю степень, равна самому
числу. Например, $153 = 1^3 + 5^3 + 3^3$. Написать программу,
находящую все трёхзначные числа Армстронга.

\task Дано число $n$. Написать программу, находящую число с наибольшей
суммой делителей, не превосходящее $n$.

\task Как показал индийский математик С.~Рамануджан, 1729 — наименьшее
число, представимое двумя различными способами в виде суммы кубов двух
натуральных чисел $x$ и $y$ ($x\geqslant y\geqslant 0$).  То есть,
$x^3+y^3=1729$. Написать программу, находящую обе пары $x$ и $y$.

\task Два натуральных числа называются дружественными, если каждое из
них равно сумме всех делителей другого (само число в качестве делителя
не рассматривается). Написать программу, находящую все пары
дружественных чисел, меньших 50000.

\task Число называется совершенным, если оно равно сумме своих
делителей за исключением его самого. Например, $6=1+2+3=1\cdot
2\cdot3$. Древним грекам были известны только первые 4 совершенных
числа. Написать программу, находящую их.
