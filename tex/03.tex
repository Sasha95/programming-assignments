\section{Циклы}

\subsection{Цикл с параметром}

\task На счёте в банке лежит $S~\textup{руб}$. Через каждый месяц
размер вклада увеличивается на $p\,\%.$ Написать программу, выводящую
таблицу с суммой вклада в каждом из следующих 12~месяцев.

\task Последовательность чисел $a_0, a_1, a_2, \dots$ образуется по
закону:
\begin{align*}
a_0 &= 1,\\
a_k &= ka_{k-1} + \frac1k.
\end{align*}
Написать программу, вычисляющую $a_n$ для заданного номера $n.$

\task Начав тренировки спортсмен в первый день пробежал
$L~\textup{км}.$ Каждый следующий день он увеличивал пробег на $p\,\%$
от пробега предыдущего дня. Написать программу, определяющую суммарный
пробег за $N$~дней.

\task Последовательность Фибоначчи определяется следующим образом:
\begin{align*}
a_0 &= 1,\\
a_1 &= 1,\\
a_k &= a_{k-1} + a_{k-2}.
\end{align*}
Написать программу, выводящую первые $n$ членов последовательности.

\task Дано натуральное число $n$. Написать программу, вычисляющую
сумму
\[
n^2+(n+1)^2+\ldots+(2n)^2.
\]

\task Написать программу, вычисляющую сумму
\[
1+\frac{1}{2}+\frac{1}{3}+\ldots+\frac1n
\]
для заданного $n.$

\task Написать программу, вычисляющую сумму
\[
\frac12 + \frac23 + \ldots + \frac{n}{n+1}
\]
для заданного $n.$

\task Известно сопротивление каждого из элементов электрической
цепи. Все элементы соединены параллельно. Написать программу,
вычисляющую общее сопротивление цепи.

\task Двойной факториал числа определяется следующим образом:
\[
n!! =
\begin{cases}
  1 \cdot 3 \cdot 5 \cdot \ldots \cdot n, &\textup{если }n\textup{ нечётное},\\
  2 \cdot 4 \cdot 6 \cdot \ldots \cdot n, &\textup{если }n\textup{ чётное}.
\end{cases}
\]
Написать программу, вычисляющую двойной факториал для заданного
натурального $n$.

\task Написать программу, вычисляющую для заданного $n$ значение суммы
\[
1+\frac1{1!} + \frac{1}{2!} + \ldots + \frac1{n!}.
\]

\subsection{Цикл с условием}

\task Дано натуральное число. Написать программу, находящую
арифметическое среднее его цифр.

\task Дано действительное число $a.$ Написать программу, находящую
такое наименьшее число $n,$ что
\[
1 + \frac12 + \frac13 + \ldots + \frac1n > a.
\]

\task Написать программу, находящую наибольший общий делитель двух
чисел с помощью алгоритма Евклида.

\task Некоторая последовательность задана рекуррентным соотношением:
\begin{align*}
  a_0 &= 1,\\
  a_k &= \cos a_{k-1}.
\end{align*}
Дано малое действительное число $\varepsilon > 0$. Написать программу, находящую первый член последовательности, для которого $|a_k - a_{k-1}| < \varepsilon.$

\task Дано натуральное число. Написание программы, проверяющей,
расположены ли цифры в нём по возрастанию слева направо.

\task Дана последовательность чисел $a_1, a_2, \ldots, a_n.$ Написать
программу, проверяющую, является ли последовательность упорядоченной
по возрастанию. Если это не так, то вывести номер первого числа,
нарушающего упорядоченность.

\task Написать программу, находящую наименьший делитель заданного
числа, отличный от $1.$

\task Написать программу, находящую наибольшую и наименьшую цифры
заданного натурального числа.

\task Дано натуральное число $n$ и цифра $k.$ Написать программу,
находящую номер первого вхождения цифры в число. (Позиции
отсчитываются справа налево начиная с 0.)

\task Дано натуральное число. Написать программу, проверяющую является
ли она палидромом. (Число называется палидромом, если оно записывается
одинаково слева направо и справа налево.)

\subsection{Вложенные циклы}

\task

\task

\task

\task

\task

\task

\task

\task

\task

\task

\subsection{Прерывание итераций}

\task

\task

\task

\task

\task

\task

\task

\task

\task

\task
