\section{LINQ}

\subsection{Простые выражения}

Увеличить все элементы коллекции на единицу используя LINQ.

Решение: для решения задачи используем следующий LINQ запрос:

from контейнер in коллекция select ++контейнер; //перебираем все
элементы коллекции и выбираем в новую коллекцию элементы увеличенные
на единицу.

Пример программы:

%% using System;
%% using System.Collections.Generic;
%% using System.Linq;

%% namespace test
%% {
%%     class Program
%%     {
%%         public static void Main(string[] args)
%%         {
%%             List<int> lint = new List<int>{1,2,3,4,23,6,7,8,9,5,4,36,0};
%%             var temp = from x in lint select ++x;
%%             foreach (var element in temp)
%%             {
%%                 Console.WriteLine(element);
%%             }
%%             Console.ReadKey(true);
%%         }
%%     }
%% }

\task Удвоить все элементы коллекции используя LINQ.

\task Найти корень квадратный из каждого элемента коллекции используя
LINQ.

\task В некоторой коллекции заданы углы, найти косинусы этих углов
используя LINQ.

\task Выбрать из коллекции абсолютные значения заданных в ней числовых
значений используя LINQ.

\task В некоторой коллекции заданы радиусы окружностей, найти для
каждой окружности площадь используя LINQ.

\task В некоторой коллекции заданы углы в градусах, найти значения
этих углов в радианах используя LINQ.

\task В некоторой коллекции заданы значения расстояния в метрах
перевести их в дюймы (1 дюйм = 2.54 см) используя LINQ.

\task Возвести каждый элемент коллекции в 5-ую степень используя LINQ.

\task Извлечь из каждого элемента коллекции корень 7-ой степени
используя LINQ.

\task Изменить знаки всех элементов коллекции на противоположные
используя LINQ.


\subsection{Выражения с условием}

Выбрать из коллекции все элементы, модуль которых больше десяти.

Решение: для решение задачи используем следующий LINQ запрос :

from контейнер in коллекция where булево выражение select контейнер;
// этот запрос работает следующим образом: перебираются все элементы
коллекции и если они удовлетворяют условию (булево выражение должно
содержать условие согласно которому производится выборка) добавляются
в результирующую коллекцию.

Пример программы:

%% using System;
%% using System.Collections.Generic;
%% using System.Linq;

%% namespace test
%% {
%%     class Program
%%     {
%%         public static void Main(string[] args)
%%         {
%%             List<int> lint = new List<int>{1,2,3,4,23,6,7,8,9,5,4,36,0};
%%             var temp = from x in lint where Math.Abs(x)>10 select x;
%%             foreach (var element in temp)
%%             {
%%                 Console.WriteLine(element);
%%             }
%%             Console.ReadKey(true);
%%         }
%%     }
%% }

\task В некоторой коллекции заданы радиусы окружностей, выбрать только
те, площади окружностей которых не меньше 4используя LINQ.

\task Задана коллекция углов заданных в градусах, выбрать только
острые углы, используя LINQ.

\task В некоторой коллекции заданы углы в радианах, выбрать те,
косинусы которых неотрицательны используя LINQ.

\task Выбрать из коллекции только целые двузначные числа используя
LINQ.

\task Выбрать из коллекции только четные элементы используя LINQ.

\task Выбрать из коллекции только нечетные элементы используя LINQ.

\task Выбрать из коллекции все неотрицательные значения используя
LINQ.

\task Выбрать из коллекции только числа являющиеся полными
квадратами(полные квадраты это числа арифметический корень из которых
есть целое число) используя LINQ.

\task Выбрать из коллекции только числа кратные семи используя LINQ.

\task В коллекции заданы значения углов в радианах, выбрать только
тупые углы, используя LINQ.


\subsection{Поиск в полях класса}

Дана коллекция объектов (объекты содержат поле Id). Необходимо выбрать
все объекты у которых Id лежит в интервале 0 до 100 и отсортировать по
убыванию.

Решение: для решения используем LINQ запрос следующего вида:

%% from x in collection where x.Id < 100 && x.Id > 0 orderby x.id
%% descending select x;// orderby x.id отсортировывает результирующую
%% коллекцию по заданному полю объекта в нашем случае по полю Id, а с
%% помощью descending мы указываем что объекты должны быть отсортированы
%% по убыванию.

Пример программы:

%% using System;
%% using System.Collections.Generic;
%% using System.Linq;

%% namespace test
%% {
%%     class _object{
%%          public int id;
%%         }
%%     class Program
%%     {
        
        
%%         public static void Main(string[] args)
%%         {
%%             List<_object> lint = new List<_object>{
%%                 new _object(){ id=0},
%%                 new _object(){ id=16},
%%                 new _object(){ id=2},
%%                 new _object(){ id=42},
%%                 new _object(){ id=31},
%%                 new _object(){ id=110},
%%                 new _object(){ id=1},
%%                 new _object(){ id=22},
%%                 new _object(){ id=40},
%%                 new _object(){ id=34}
%%             };
%%             var temp = from x in lint where x.id >0 && x.id <100 orderby x.id descending select x;
%%             foreach (var element in temp)
%%             {
%%                 Console.WriteLine(element.id);
%%             }
%%             Console.ReadKey(true);
%%         }
%%     }
%% }

\task Задана коллекция объектов типа Circle и у каждого такого объекта
есть поле Rad (радиус), выбрать только окружности, радиусы которых не
превышают сорока и отсортировать по возрастанию.

\task Задана коллекция объектов типа Triangle и у каждого такого объекта
есть поля side1, side2, side3, выбрать только такие треугольники,
периметр которых не меньше пятидесяти.

\task Задана коллекция объектов типа Student и у каждого такого объекта
есть поля Height(рост в сантиметрах) и Age(возраст в годах) выбрать
только совершеннолетних студентов, рост которых не меньше
180см. Отсортировать студентов по росту в порядке возрастания.

\task Задана коллекция объектов типа Cat и у каждого такого объекта есть
поле Color принимающее значения от 1 до 5 (1 – черный кот, 2-белый
кот, 3-рыжий кот, 4 – пепельный кот, 5- полосатый кот), выбрать только
рыжих и полосатых котов. В результирующей коллекции сначала должны
быть все рыжие коты, а после полосатые.

\task Задана коллекция объектов типа Dog и у каждого такого объекта есть
поле WantToEat(поле типа boolean) выбрать только голодных собак.

\task Задана коллекция объектов типа Apple и у каждого такого объекта
есть поле Price(цена) выбрать только те яблоки, стоимость которых не
превышает 5.50. Отсортировать по цене в порядке убывания.

\task Задана коллекция объектов типа Computer и у каждого такого объекта
есть поля Price(цена) и Year(год выпуска) выбрать только компьютеры,
выпущенные не раньше 2010 года и стоимость которых не превышает
550\$.Отсортировать по цене в порядке убывания.

\task Задана коллекция объектов типа Car и у каждого такого объекта есть
поле TopSpeed(максимальная скорость) выбрать только автомобили
способные развивать скорость более 250км/ч. Отсортировать по убыванию.

\task Задана коллекция объектов типа Hat и у каждого такого объекта есть
поле Size(размер ) выбрать только шляпы 4-го размера.

\task Задана коллекция объектов типа Tank и у каждого такого объекта
есть поля MaxM(максимальная грузоподъемность) и Calibr(калибр
основного орудия) выбрать только танки с максимальной
грузоподъемностью не более 40 тонн и орудием, калибр которого не
превышает 125мм.Отсортировать по максимальной грузоподъемности в
порядке возрастания.
