\section{Файлы и сериализация}

\subsection{Файлы}

\task В текстовом файле хранится таблица действительных чисел. В
каждой строке записаны три числа, разделённые символом «;». Написать
программу, находящую для каждого столбца сумму и арифметическое
среднее.

\task Написать программу, оценивающую информационную энтропию в
указанном файле.\index{Энтропия информационная}

Для этого она должна просмотреть файл побайтово и составить таблицу
частот $p_k$ каждого значения байта ($k$ изменяется от 0 до
255). После этого можно вычислить энтропию по формуле
\[
H = - \sum_{k=0}^{255} p_k \log_2 p_k.
\]

Информационная энтропия в данном случае будет показывать оценку
количества информации, приходящейся на один байт файла. Чем она
меньше, тем сильнее можно сжать файл.

В приведённой формуле в качестве $p_k$ должны использоваться
вероятности. Так как они, вообще говоря, неизвестны, вместо них
используются частоты. Частота байта — это отношение количества его
вхождений к общему количеству байтов в файле.

\task В текстовом файле записана информация о продолжительности
телефонных звонков.  Информация о каждом звонке записана в отдельной
строке в формате «Фамилия:Продолжительность». Продолжительность
указана в секундах.

Написать программу, находящую суммарную продолжительность разговоров
для каждого из абонентов.

\task Написать функции для сохранения в двоичный файл и чтения из него
двумерного массива целых чисел. В начало файла должны записываться
размеры массива, а затем в двоичном виде элементы.

Элементы записываются построчно, то есть, вначале в файл помещаются
все элементы первой строки массива, затем второй и так далее.

Функция для записи должна принимать имя файла и массив, функция для
чтения — только имя файла и возвращать полученный массив.

Написать программу, использующую эти функции.

\task Написать программу, шифрующую файл и записывающую результат в
новый файл. Шифрование выполнить прибавлением по модулю 2
(«исключающее или», XOR) к каждому байту исходного файла заданного
однобайтового числа.

Этот метод называется гаммированием.\index{Гаммирование} В случае
наложения всего одного байта он крайне ненадёжен и не должен
использоваться для обеспечения безопасности данных.

\task Написать программу, удваивающую каждый байт указанного двоичного
файла с насыщением. Насыщение\index{Насыщение} означает, что если
удвоенное значение превышает 255 (максимальное значение для байта), то
результат равен 255.

\task Написать программу, выводящую те строки текстового файла, в
которых встречается указанное слово. Перед выводимой строкой указать
её номер. В конце вывести общее количество вхождений слова.

\task Написать программу, просматривающую указанный файл побайтово и
выводящую на экран 10 самых частых значений байтов с указанием
частоты.

\task Написать программу, выводящую отсортированный список слов без
повторений, встречающихся в указанном текстовом файле.

Считать словом последовательность букв, регистр символов не учитывать
(для этого перевести все слова в нижний регистр). Остальные символы в
файле игнорировать.

\task Написать программу, выводящую таблицу частот русских букв (без
учёта регистра) в заданном текстовом файле. Таблицу упорядочить по
убыванию частоты.

Частотой буквы считать отношение количества её вхождений к общему
количеству букв.


\subsection{Файловая система}

\task Написать программу, дописывающую к имени всех файлов с
расширением «*.txt» слева через пробел дату создания.

Дата должна быть записана в соответствии со стандартом
ISO~8601\index{Стандарт ISO~8601}: год, месяц и день через
дефисы. Например: «2016-01-18» для 18~января 2016~года. Этот способ
удобен тем, что позволяет сортировать даты естественным образом.

\task Написать программу, выводящую на экран список файлов с
расширением «.txt» в заданной папке, в которых содержится указанное
слово.

\task Написать программу, выводящую на экран все файлы в указанной
папке с указанием размера. Также для каждого файла указать его долю в
общем объёме.

\task Написать программу, которая объединяет содержимое всех файлов с
именем, заканчивающимся на «.part.txt» и записывающая результат в файл
с указанным именем. Файлы объединяются в алфавитном порядке.

\task Написать программу, выводящую список файлов, одновременно
присутствующих в двух указанных папках. Сравнивать файлы по
содержимому не требуется, достаточно искать файлы с одинаковыми
именами.

\task Дан файл со списком имён файлов (по одному на строку). Написать
программу, проверяющую, присутствуют ли файлы из списка в указанной
папке. Имена отсутствующих файлов вывести на экран.

\task Написать программу, перемещающему каждый файл в папке в подпапку
с именем, равным первому символу имени файла (без учёта
регистра). Например, все файлы с именем, начинающимся с буквы «A»
будут перемещены в папку «A» и т.~д. (Папки должны создаваться
программой при необходимости.)

\task Написать программу, выводящую на экран список всех файлов с
расширением «.txt» в указанной папке. После каждого имени также должна
выводиться первая строка файла.

\task Написать программу, выводящую список файлов в указанной папке,
созданных раньше указанной даты.

\task Написать программу, перемещающую каждый файл в указанной папке в
папку с именем, равным дате создания файла. Папки должны создаваться
программой при необходимости.

Дата должна быть записана в соответствии со стандартом
ISO~8601\index{Стандарт ISO~8601}: год, месяц и день через
дефисы. Например: «2016-01-18» для 18~января 2016~года. Этот способ
удобен тем, что позволяет сортировать даты естественным образом.


\subsection{XML и сериализация}

\task Написать программу, запрашивающую список граждан (определяется
фамилий, именем, отчеством и датой рождения) и записывающую его в
файл в формате XML.

Написать программу, выводящую список из XML-файла.

\task Написать программу, получающую список файлов в указанной папке и
сохраняющие их имена и даты создания в файл в формате XML.

Написать программу, выводящую список из XML-файла.

\task Написать программу, запрашивающую с клавиатуры фамилии и номера
телефонов и сохраняющую их в файл в формате XML.

Написать программу для поиска номера телефона по фамилии в XML-файле.

\task Дан файл, хранящий переводы английских слов. Каждая пара перевод
— слово занимает одну строку и записана в формате «слово - перевод».

Написать программу сохраняющую информацию о переводах в файл в формате
XML.

\task Написать программу, запрашивающую с клавиатуры информацию о
фильмах (название, режиссёр, год выхода) и сохраняющую её в файл в
формате XML.

Написать программу, выводящую информацию о фильмах, сохранённую в
XML-файл.

\task Дан файл в формате CSV. В первых трех столбцах записаны фамилия,
имя и отчество, а в четвёртом — адрес электронной почты. Написать
программу для преобразования файла в формат XML.

В формате CSV данные записываются построчно и разделяются запятыми. В
случае, если запятая уже используется в качестве разделителя десятичной
части числа (как в русском языке), используют точку с запятой.

\task Написать программу, запрашивающую информацию о книгах (автор,
название и год издания) и сохраняющую её в файл в формате
XML.

Написать программу, выводящую информацию о книгах, сохранённую в
XML-файл.

\task Написать программу, запрашивающую информацию о группе (номер,
год поступления и фамилии студентов) и сохраняющую её в файл в формате
XML.

Написать программу, выводящую информацию о группе, сохранённую в
XML-файл.

\task Дан файл, хранящий информацию о докладах на
конференции. Информация о каждом докладе занимает одну строку и
записана в формате «Фамилия: Тема (Дата)».

Написать программу для преобразования файла в формат XML.

\task Написать программу, запрашивающую информацию о студентах
(фамилия и средний балл) и сохраняющую её в файл в формате XML.

Написать программу, принимающую на вход XML-файл и выводящую фамилии
трёх лучших студентов.
