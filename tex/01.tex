\section{Линейные алгоритмы}

\subsection{Арифметические выражения}

\task Даны прямоугольник с длиной $a$ и высотой $b$. Написать
программу, вычисляющую площадь прямоугольника, длина коротого больше в
$4$ раза, а ширина меньше в $1{,}5.$

\task Велосипедист движется равномерно прямолинейно со скоростью
$v~\frac{\textup{м}}{\textup{с}}.$ Написать программу, вычисляющую
время в минутах, за которое он пройдёт $s~\textup{км}.$

\task Написать программу, вычисляющую сумму членов бесконечно
убывающей геометрической прогрессии, первый член которой $b$, а разность
$q$.

\task Даны два тела с массами $m_1$ и $m_2$, расположенные на
расстоянии $d$ друг от друга. Найти силу их взаимного
притяжения. (Гравитационная постоянная $\gamma=6{,}67\times
10^{-11}~\frac{\textup{м}^3}{\textup{кг}\cdot\textup{с}^2}$.)

\task Тело массой $m$ подброшено вертикально с начальной скоростью
$v$. Написать программу, вычисляющую высоту тела в момент времени $t$.
(Ускорение свободного падения
$g=9{,}81~\frac{\textup{м}}{\textup{с}^2}$.)

\task Длина комнаты без окон равна $a$, ширина — $b$. В комнате есть
дверь площадью $w\times h$. (Все размеры даны в метрах.) Написать
программу, вычисляющую общую площадь стен.

\task Написать программу, вычисляющую сумму первых $n$ ($n>3$) членов
арифметической прогрессии, если известны первый член $a_1$ и третий
$a_3$.

\task Автомобиль двигаясь прямолинейно и равноускоренно за
$t~\textup{с}$ проехал $s~\textup{м}.$ Написать программу, вычисляющую
его ускорение, если начальная скорость была нулевой.

\task На проводник длины $L$ с током силой $I$ действует магнитное
поле. Вектор магнитной индукции перпендикулярен проводнику, и его
модуль равен $B$. Написать программу, вычисляющую силу действующую на
проводник.

\task Дан прямоугольный треугольник с катетами $a$ и $b$, лежащими на
осях координат. Найти координаты его центра тяжести.


\subsection{Работа с объектом Math}

\task Написать программу, вычисляющую площадь кольца со внутреним и
внешним радиусами, равными $r$ и $R$ соответственно.

\task Дан треугольник со сторонами $a$, $b$ и $c$. Написать программу,
определяющую угол между сторонами $a$ и $b$.

\task Написать программу, вычисляющую расстояние между точками $(x_1,
y_1)$ и $(x_2, y_2)$.

\task Дан треугольник со сторонами $a$, $b$ и $c$. Написать программу,
вычисляющую его площадь.

\task Написать программу, определяющую угол между векторами с
координатами $(a_x, a_y)$ и $(b_x, b_y)$.

\task Дана длина окружности $l$. Найти площадь круга, ограниченного
ей.

\task Даны катеты $a$ и $b$ прямоугольного треугольника. Найти радиус
вписанной окружности.

\task Написать программу, находящую значение выражения
$\left|\frac{x+y}2\sqrt{\frac{\ln (x+y)}{xy}}\right|$ для заданных $x$
и $y$.

\task Написать программу, вычисляющую расстояние до линии горизонта от
точки, расположенной на высоте $h$ над поверхностью Земли. (Считать
Землю идеальной сферой с радиусом $R=6350~\textup{км}$.)

\task Известно расстояние $D$ между двумя наблюдательными пунктами и
углы $\alpha_1$ и $\alpha_2$, под которыми с них видна цель. Углы
определяются между направлением на цель и направлением на другой
наблюдательный пункт. Написать программу, вычисляющую расстояния от
цели до наблюдательных пунктов.


\subsection{Целочисленная арифметика}

\task Написать программу меняющую местами две первые цифры с двумя
последними в заданном четырёхзначном числе и выводящую сумму всех
цифр.

\task Дано пятизнеачное число. Составить программу, вычисляющую сумму
цифр стоящих на нечётных позициях.

\task Дан размер документа в байтах. Написать программу, вычисляющую
количество полных килобайтов и мегабайтов.

\task Дана масса тела в граммах ($m>1000$). Написать программу,
вычисляющую количество целых килограммов и центнеров.

\task Написать программу, вычисляющую целую часть от деления
числа,составленного из первых трёх цифр заданного пятизначного числа,
на цисло составленное из оставшихся двух цифр.

\task Написать программу, находящую сумму и произведение цифр
четырёхзначного числа, а так же вычисляющую разность этих двух цифр.

\task С начала суток прошло $s$ секунд. Составить программу, вычисляющую
сколько прошло полных минут и часов.

\task Написать программу, вычисляющую сумму и произведение цифр
пятизначного числа и вычисляющую остаток от деления произведения на
сумму.

\task Дано четырёхзначное число. Написать программу, составляющую
новое число, первая цифра которого целая часть от деления первой цифры
на вторую, а вторая остаток от деления третьеё цифры на четвёртую.

\task Сколько кубиков с длиной ребра $h$ можно поместить в коробку с
размерами $a\times b\times c.$

\subsection{Операции ввода и вывода}

\task Даны стороны прямоугольника. Написать программу, вычисляющую
площадь $S$ и периметр $P$ данного прямоугольника. Ответ дать в виде:
«Площадь прямоугольника S см2. Периметр прямоугольника P см.» При
запросе значений с клавиатуры выводить подсказку.

\task Даны два числа $a$ и $b$. Написать программу, вычисляющую среднее
арифметическое $k$ этих двух чисел. Ответ дать в виде: «Среднее
арифметическое чисел а и b равно k.»

\task Даны два числа $k$ и $m$. Написать программу, вычисляющую среднее
геометрическое n этих двух чисел. Ответ вывести в виде: «Среднее
геометрическое чисел k и m равно n .»

\task Даны числа $a$ и $b$. Написать программу, вычисляющую сумму и
произведеение квадратов этих чисел. Ответ дать, используя один
оператор вывода.

\task Вычислить значения выражения $|x^6+ln(x)+10|$ для $x=k$, $x=l$,
$x=m$. Вывести ответ одним выражением вида: «Для х=k значение выражения
... Для x=m значение выражения ... Для x=l значение выражения ...»

\task Дано выражение $ln(m+10)+|m+m^10|$. Вычислить значение этого
выражения для трёх различных значений вводимых с клавиатуры. Ответ
дать аналогично ответу в задаче 5.

\task Даны катет и гипотенуза прямоугольного треугольника. Написать
программу, вычисляющую второй катет и площадь треугольника. Оформить
ввод с комментариями для пользователя, ответ дать, изпользуя один
оператор вывода.

\task Дана площадь круга. Написать программу вычисляющую радиус и
длину окружности, ограничивающей этот круг. Ответ дать, используя один
оператор вывода.

\task Дана площадь поверхности куба. Написать программу, вычисляющую
длину ребра и объём данного куба. Ответ дать в виде: «Длина ребра куба
... Объём куба...»

\task Даны два числа $n$ и $k$. Написать программу вычисляющую их сумму s
и значение выражения $s+ n/k+s^2+n^2+(n-k)^2$. Ответ дать в
виде: «Сумма чисел n и k равна s. Значение данного выражения равно ...»
