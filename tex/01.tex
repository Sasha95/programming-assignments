\section{Линейные алгоритмы}


\subsection{Арифметические выражения}

\task Дан прямоугольник с длиной $a$ и высотой $b$. Написать
программу, вычисляющую площадь другого прямоугольника, длина которого
больше в $m$ раз, а высота меньше в $n.$

\task Велосипедист движется равномерно и прямолинейно со скоростью
$v~\frac{\textup{м}}{\textup{с}}.$ Написать программу, вычисляющую
время в минутах, за которое он пройдёт $s~\textup{км}.$

\task Даны два тела с массами $m_1$ и $m_2$, расположенные на
расстоянии $d$ друг от друга. Найти силу их взаимного
притяжения. (Гравитационная постоянная $\gamma\approx 6{,}67\times
10^{-11}~\frac{\textup{м}^3}{\textup{кг}\cdot\textup{с}^2}$.)

\task Написать программу, вычисляющую сумму членов бесконечно
убывающей геометрической прогрессии, первый член которой равен $b$, а
разность — $q$.

\task Тело подброшено вертикально с начальной скоростью $v$. Написать
программу, вычисляющую высоту тела в момент времени $t$.  (Ускорение
свободного падения $g\approx 9{,}81~\frac{\textup{м}}{\textup{с}^2}$,
сопротивлением воздуха пренебречь.)

\task Длина комнаты без окон равна $a$, ширина — $b$, высота — $c$. В
комнате есть дверь площадью $w\times h$. (Все размеры даны в метрах.)
Написать программу, вычисляющую общую площадь стен.

\task Написать программу, вычисляющую сумму первых $n$ ($n>3$) членов
арифметической прогрессии, если известны первый член $a_1$ и третий
$a_3$.

\task Автомобиль двигаясь прямолинейно и равноускоренно за
$t~\textup{с}$ проехал $s~\textup{м}.$ Написать программу, вычисляющую
его ускорение, если начальная скорость была нулевой.

\task На проводник длины $L$ с током силой $I$ действует магнитное
поле. Вектор магнитной индукции перпендикулярен проводнику, и его
модуль равен $B$. Написать программу, вычисляющую силу, действующую на
проводник.

\task Дан прямоугольный треугольник с катетами $a$ и $b$, лежащими на
осях координат. Катет $a$ лежит на оси абсцисс, а катет $b$ — на оси
ординат. Найти координаты его центра тяжести.


\subsection{Класс Math}

\task Написать программу, вычисляющую площадь кольца со внутренним и
внешним радиусами, равными $r$ и $R$ соответственно.

\task Дан треугольник со сторонами $a$, $b$ и $c$. Написать программу,
вычисляющую угол между сторонами $a$ и $b$.

\task Написать программу, вычисляющую расстояние между точками $(x_1,
y_1)$ и $(x_2, y_2)$.

\task Дан треугольник со сторонами $a$, $b$ и $c$. Написать программу,
вычисляющую его площадь.

\task Написать программу, вычисляющую угол между векторами с
координатами $(a_x, a_y)$ и $(b_x, b_y)$.

\task Дана длина окружности $l$. Найти площадь круга, ограниченного
ей.

\task Даны катеты $a$ и $b$ прямоугольного треугольника. Найти радиус
вписанной окружности.

\task Написать программу, находящую значение выражения
$\left|\frac{x+y}2\sqrt{\frac{\ln (x+y)}{xy}}\right|$ для заданных $x$
и $y$.

\task Написать программу, вычисляющую расстояние до линии горизонта от
точки, расположенной на высоте $h$ над поверхностью Земли. (Считать
Землю идеальной сферой с радиусом $R=6350~\textup{км}$.)

\task Известно расстояние $D$ между двумя наблюдательными пунктами и
углы $\alpha_1$ и $\alpha_2$, под которыми с них видна цель. Углы
определяются между направлением на цель и направлением на другой
наблюдательный пункт. Написать программу, вычисляющую расстояния от
цели до наблюдательных пунктов.


\subsection{Целочисленная арифметика}

\task Написать программу, меняющую местами две первые цифры с двумя
последними в заданном четырёхзначном числе.

\task Дано пятизначное число. Составить программу, вычисляющую сумму
цифр стоящих на нечётных позициях. Позиции нумеруются справа налево
начиная с 1.

\task Дан размер документа в байтах. Написать программу, вычисляющую
количество полных килобайтов и мегабайтов.

\task Дана масса тела в граммах ($m > 1000~\textup{г}$). Написать
программу, выводящую количество целых килограммов и центнеров.

\task Написать программу, вычисляющую целую часть от деления
числа, составленного из первых трёх цифр заданного пятизначного числа,
на число, составленное из оставшихся двух цифр.

\task Написать программу, находящую сумму и произведение цифр
четырёхзначного числа.

\task Дано целое число $k$ ($0\leqslant k \leqslant 26$). Вывести его
представление в троичной системе.

\task Дано четырёхзначное число. Его цифры суммируются, после чего эта
операция применяется к результату, пока не останется одна
цифра. Написать программу, находящую эту цифру.

\task Сколько кубиков с длиной ребра $h$ можно поместить в коробку с
размерами $a\times b\times c.$

\task С начала суток до некоторого момента прошло
$s~\textup{с}$. Написать программу, выводящую время в этот момент с
точностью до минут.


\subsection{Операции ввода и вывода}

\task Дан угол в радианах. Написать программу, переводящую его в
градусы, минуты и секунды (градусы и минуты — целые числа).  Ответ
вывести в виде: «Угол $a~\textup{рад}$ равен $d°$ $m′$ $s″$». Вместо
буквенных обозначений должны стоять конкретные числа с точностью до
2-го знака после запятой. Перед запросом ввода с клавиатуры выводить
подсказку. (Код символа градуса в Unicode — 00B0.)

\task Даны числа $a$, $b$ и $c$. Написать программу, вычисляющее их
среднее гармоническое $g = \frac3{\frac1a +\frac1b + \frac1c}$. Ответ
вывести в виде: «Среднее гармоническое чисел $a$, $b$ и $c$ равно
  $g$.». Вместо буквенных обозначений должны стоять конкретные числа с
точностью до 2-го знака после запятой.  Перед запросом ввода с
клавиатуры выводить подсказку.

\task Дан угол в градусах. Написать программу, переводящую его в
деления угломера (единица измерения углов, принятая в
артиллерии). 6000 делений угломера составляют 360°. Последние две
цифры обычно отделяются дефисом. Например, 750 делений угломера
записываются как «7-50». Ответ вывести в виде: «Угол $d°$ равен
  $a$-$b$.». Вместо буквенных обозначений должны стоять конкретные
целые числа. Перед запросом ввода с клавиатуры выводить
подсказку. (Код символа градуса в Unicode — 00B0.)

\task Даны два точечных заряда $q_1$ и $q_2$, расположенные на
расстоянии $d$ друг от друга. Написать программу, вычисляющую силу их
взаимного притяжения. Коэффициент пропорциональности $k=\frac1{4\pi
  \varepsilon_0},$ где $\varepsilon_0\approx 8{,}85\times
10^{-12}~\frac{\textup{Ф}}{\textup{м}}$. Ответ вывести в виде: «Сила
  притяжения между зарядами $q_1~\textup{Кл}$ и $q_2~\textup{Кл}$,
  находящихся на расстоянии $d~\textup{м}$, равна
  $F~\textup{Н}$.». Вместо буквенных обозначений должны стоять
конкретные числа с точностью до 2-го знака после запятой. Перед
запросом ввода с клавиатуры выводить подсказку.

\task Написать программу, вычисляющую площадь правильного
$n$-угольника, вписанного в окружность радиуса $R$.  Ответ вывести в
виде: «Площадь правильного $n$-угольника, вписанного в окружность
  радиуса $R$, равна $S$.». Вместо буквенных обозначений должны стоять
конкретные числа с точностью до 2-го знака после запятой. Перед
запросом ввода с клавиатуры выводить подсказку.

\task Написать программу, находящую для некоторого $x$ приближённое
значение $\sin x$ по формуле $\sin x\approx x - \frac{x^3}6$ и
вычисляющую абсолютную погрешность результата. Ответ вывести в виде:
«Значение $\sin x$ приближённо равно $y$ (погрешность —
  $e$).». Вместо буквенных обозначений должны стоять конкретные числа
с точностью до 2-го знака после запятой. Перед запросом ввода с
клавиатуры выводить подсказку.

\task Написать программу, вычисляющую для некоторого момента времени,
заданного часами и минутами (целые числа), угол в градусах между
часовой и минутной стрелками. Ответ вывести в виде: «В момент $h$:$m$
  угол между стрелками равен $d°$.». Вместо буквенных обозначений
должны стоять конкретные целые числа. Перед запросом ввода с
клавиатуры выводить подсказку. (Код символа градуса в Unicode — 00B0.)

\task Написать программу вычисляющую сумму вклада через
$n~\textup{лет}$ при $p\,\%$ годовых, если первоначальный вклад был
равен $S_0~\textup{руб.}$ Ответ вывести в виде: «При первоначальном
  вкладе $S_0~\textup{руб.}$ и $p\,\%$ годовых сумма вклада составит
  $S_n~\textup{руб.}$.». Вместо буквенных обозначений должны стоять
конкретные числа с точностью до 2-го знака после запятой. Перед
запросом ввода с клавиатуры выводить подсказку.

\task Написать программу, переводящую вес в килограмах в фунты и
унции. $1~\textup{фунт} \approx 0{,}454~\textup{кг},$ при этом в
1~фунте 16~унций. Ответ вывести в виде: «$m~\textup{кг} \approx
  p~\textup{фунтов}\ o~\textup{унций}$». Вместо буквенных обозначений
должны стоять конкретные целые числа. Перед запросом ввода с
клавиатуры выводить подсказку.

\task Написать программу, переводящую сумму в рублях в евро для
указанного обменного курса.  Ответ вывести в виде: «$R~\textup{руб.}$
  = $E~\textup{евро}$.». Вместо буквенных обозначений должны стоять
конкретные числа с точностью до 2-го знака после запятой. Перед
запросом ввода с клавиатуры выводить подсказку.
