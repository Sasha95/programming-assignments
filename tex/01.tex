\section{Линейные алгоритмы}

\subsection{Арифметические выражения}

\task Даны стороны a и b прямоугольника. Написать программу,
вычисляющую площадь прямоугольника, длина коротого больше в 4 раза, а
ширина меньше в 1,5.

\task Велосипедист движется равномерно прямолинейно со скоростью V
м/с.

a)Написать программу, вычисляющую время,за которое он пройдёт S
км.(Ответ дать в секундах).

b)Написать программу, вычисляющую расстояние, которое он пройдёт за
время t мин.(Ответ дать в километрах.)

\task Написать программу, вычисляющую сумму членов бесконечно
убывающей геометрической прогрессии, первый член которой b, а разность
q, и определяющую 3-й член этой прогрессии.

\task Написать программу, вычисляющую площадь треугольника с
основанием k и высотой h.

\task Написать программу, определяющую потенциальную энергию предмета
массой m, находящегося на высоте H.

\task Написать программу, вычисляющую площадь круга, диаметр которого
D.

\task Написать программу, вычисляющую среднее арифметическое 4 чисел.

\task Автомобиль двигаясь прямолинейно равноускоренно за t секунд
проехал S метров. Написать программу, вычисляющую его ускорение, если
начальная скорость равна 0.

\task Даны 4 целых числа. Написать программу для нахождения среднего
арифметического этих чисел.

\task Написать программу, вычисляющую сумму первых 10 членов
арифметической прогрессии с шагом 5, если её первый член 4.


\subsection{Работа с объектом Math}

\task Написать программу, находящую значение выражения
$x+(x^5-10)*e^4/(x^6-10*x+log10(x-5))$.

\task Написать программу, вычисляющую
$|x-5+(10*x-ln15)/(3*x^2-5*x+1)|$. (x-произвольное число).

\task Даны 3 стороны треугольника: a, b, c. Написать программу,
определяющую угол между сторонами a и b, используя теорему косинусов.

\task Написать программу, находящую среднее геометрическое чисел a и
b.

\task Написать программу, вычисляющую расстояние между точками
А(x1,y1) и B(x2,y2).

\task Написать программу, вычисляющую площадь треугольника со
сторонами a, b, c, по формуле Герона.

\task Написать программу, определяющую угол между векторами a(a1,a2) и
b(b1,b2).

\task Написать программу, вычисляющую диагональ прямоугольного
параллелепипеда с длинной a, шириной b и высотой c.

\task Написать программу для нахождения длины вектора AB(x,y).

\task Написать программу, находящую значение выражения
$|x-y+sqrt((x-y^2)/(x^2-y-x*y*sin(y)))|$.


\subsection{Целочисленная арифметика}

\task Написать программу меняющую местами две первые цифры с двумя
последними в заданном четырёхзначном числе и выводящую сумму всех
цифр.

\task Дано пятизнеачное число. Составить программу, вычисляющую сумму
цифр стоящих на нечётных позициях.

\task Дан размер документа в байтах. Написать программу, вычисляющую
количиство полных килобайтов и мегобайтов.

\task Дана масса тела в граммах (m>1000). Написать программу,
вычисляющую количество целых килограммов и центнеров.

\task Написать программу, вычисляющую целую часть от деления
числа,составленного из первых трёх цифр заданного пятизначного числа,
на цисло составленное из оставшихся двух цифр.

\task Написать программу, находящую сумму и произведение цифр
четырёхзначного числа, а так же вычисляющую разность этих двух цифр.

\task С начала суток прошло s секунд. Составить программу, вычисляющую
сколько прошло полных минут и часов.

\task Написать программу, вычисляющую сумму и произведение цифр
пятизначного числа и вычисляющую остаток от деления произведения на
сумму.

\task Дано четырёхзначное число. Написать программу, составляющую
новое число, первая цифра которого целая часть от деления первой цифры
на вторую, а вторая остаток от деления третьеё цифры на четвёртую.

\subsection{Операции ввода и вывода}

\task Даны стороны прямоугольника. Написать программу, вычисляющую
площадь S и периметр P данного прямоугольника. Ответ дать в виде:
"Площадь прямоугольника S см2. Периметр прямоугольника P см."

\task Даны два числа a и b. Написать программу, вычисляющую среднее
арифметическое k этих двух чисел. Ответ дать в виде: "Среднее
арифметическое чисел а и b равно k."

\task Даны два числа k и m. Написать программу, вычисляющую среднее
геометрическое n этих двух чисел. Ответ вывести в виде: "Среднее
геометрическое чисел k и m равно n ."

\task Даны числа a и b. Написать программу, вычисляющую сумму и
произведеение квадратов этих чисел. Ответ дать, используя один
оператор вывода.

\task Вычислить значения выражения $|x^6+ln(x)+10|$ для x=k, x=l,
x=m. Вывести ответ одним выражением вида: "Для х=k значение выражения
... Для x=m значение выражения ... Для x=l значение выражения ..."

\task Дано выражение $ln(m+10)+|m+m^10|$. Вычислить значение этого
выражения для трёх различных значений вводимых с клавиатуры. Ответ
дать аналогично ответу в задаче 5.

\task Даны катет и гипотинуза прямоугольного треугольника. Написать
программу, вычисляющую второй катет и площадь треугольника. Оформить
ввод с комментариями для пользователя, ответ дать, изпользуя один
оператор вывода.

\task Дана площадь круга. Написать программу вычисляющую радиус и
длину окружности, ограничивающей этот круг. Ответ дать, используя один
оператор вывода.

\task Дана площадь поверхности куба. Написать программу, вычисляющую
длину ребра и объём данного куба. Ответ дать в виде: "Длина ребра куба
... Объём куба..."

\task Даны два числа n и k. Написать программу вычисляющую их сумму s
и значение выражения $s+ n/k+s^2+n^2+(n-k)^2$. Ответ дать в
виде:"Сумма чисел n и k равна s. Значение данного выражения равно ..."
