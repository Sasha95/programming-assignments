\section{Компонентное программирование}

\subsection{Библиотеки классов}

\task Разработать библиотеку классов для вычисления значения
многочлена в заданной точке по списку коэффициентов. Так, список
$(1, 2, 3)$ соответствует многочлену $P(x)=1x^2+2x+3.$ Также
реализовать операции сложения и вычитания многочленов (степень может
различаться).

Разработать приложения с графическим интерфейсом и интерфейсом
командной строки, использующие эту библиотеку. Приложения должны
запрашивать входные данные и выполняемую операцию.

\task Разработать библиотеку классов для приближённого вычисления
значения производных элементарных функций. Программа должна
поддерживать дифференцирование не менее 10 функций. (Удобно хранить
элементарные функции в виде лямбда-выражений в словаре.)

Для поиска производной функции $f(x)$ в точке $x_0$ с шагом $h$ можно
воспользоваться формулой:
\[
f'(x_0) = \frac{f(x_0+h) - f(x_0-h)}{2h}.
\]

Разработать приложения с графическим интерфейсом и интерфейсом
командной строки, использующие эту библиотеку. Приложения должны
запрашивать входные данные (точку и шаг) и дифференцируемую функцию.

\task Разработать библиотеку классов для решения в целых числах
уравнений
\begin{align*}
a x + b y &= c,\\
a x^2 + b y &= c,\\
\end{align*}
где $a, b, c = \mathrm{const}.$ (Такие уравнения называются
диофантовыми.)

Решения искать в диапазонах $x_{min}\leqslant x \leqslant x_{max}$ и
$y_{min}\leqslant y \leqslant y_{max}.$

Разработать приложения с графическим интерфейсом и интерфейсом
командной строки, использующие эту библиотеку. Приложения должны
запрашивать входные данные (коэффициенты и диапазон) и решаемое
уравнение.

\task Разработать библиотеку классов для арифметических операций
(сложение, вычитание, скалярное произведение) над векторами
произвольной размерности.

Разработать приложения с графическим интерфейсом и интерфейсом
командной строки, использующие эту библиотеку. Приложения должны
запрашивать входные данные и выполняемую операцию.

\task Разработать библиотеку классов для арифметических операций над
обыкновенными дробями. Дробь задаётся числителем и знаменателем.

Разработать приложения с графическим интерфейсом и интерфейсом
командной строки, использующие эту библиотеку. Приложения должны
запрашивать входные данные и выполняемую операцию.

\task Разработать библиотеку классов для преобразования обыкновенных
дробей в непрерывные и обратно.

Непрерывная (цепная) дробь — это дробь вида
\[
[a_0; a_1, a_2, a_3, \dots] = a_0 + \frac1{a_1 + \frac1{a_2 + \frac1{a_3 + \dots}}}.
\]

Для преобразования обыкновенной дроби $x$ в непрерывную можно
воспользоваться следующей процедурой:
\begin{align*}
&a_0 = \left\lfloor x \right\rfloor,\quad x_0 = \left\{x\right\},\\
&a_1 = \left\lfloor \frac1{x_0} \right\rfloor,\quad x_1 = \left\{\frac1{x_0}\right\},\\
&a_2 = \left\lfloor \frac1{x_1} \right\rfloor,\quad x_2 = \left\{\frac1{x_1}\right\}, \dots
\end{align*}
Здесь $\{x\}$ — дробная часть $x$. Вычисления продолжаются до тех пор,
пока $x_i$ не станет равен $0$ или не будет достаточно мал (для
иррациональных чисел).

Разработать приложения с графическим интерфейсом и интерфейсом
командной строки, использующие эту библиотеку. Приложения должны
запрашивать входные данные и направление преобразования.

\task Разработать библиотеку классов для арифметических операций
(сложения, вычистания, умножения и деления) в поле Галуа $GF(4)$. Поле
$GF(4)$ — это множество из элементов ${0, 1, a, b},$ в котором
операции сложения и умножения определены следующим образом:

\begin{table}
  \parbox{.45\linewidth}{
    \centering
    \begin{tabular}{c|cccc}
      +&0&1&a&b\\
      \hline
      0&0&1&a&b\\
      1&1&0&b&a\\
      a&a&b&0&1\\
      b&b&a&1&0
    \end{tabular}
  }
  \hfill
  \parbox{.45\linewidth}{
    \centering
    \begin{tabular}{c|cccc}
      ×&0&1&a&b\\
      \hline
      0&0&0&0&0\\
      1&0&1&a&b\\
      a&0&a&b&1\\
      b&0&b&1&a
    \end{tabular}
  }
\end{table}
Вычитание и деление выполняется по таблицам. Например, $1:b = a,$ так
как $a\cdot b=1.$

Поля Галуа $GF(k)$ широко применяются в криптографии и
помехоустойчивых кодах.

Разработать приложения с графическим интерфейсом и интерфейсом
командной строки, использующие эту библиотеку. Приложения должны
запрашивать входные данные и выполняемую операцию.

\task Разработать библиотеку классов для вычисления площади общей
части, суммарной площади, а также сравнения площадей двух
прямоугольников, заданных координатами противоположных вершин.

Разработать приложения с графическим интерфейсом и интерфейсом
командной строки, использующие эту библиотеку. Приложения должны
запрашивать входные данные и выполняемую операцию.

\task Разработать библиотеку классов для сложения и вычитания угловых
мер, заданных градусами, минутами и секундами.

Разработать приложения с графическим интерфейсом и интерфейсом
командной строки, использующие эту библиотеку. Приложения должны
запрашивать входные данные и выполняемую операцию.

\task Разработать библиотеку классов для преобразования строкового
представления $n$-значного числа ($2\leqslant n \leqslant 10$) в
десятичное число и обратно.

Разработать приложения с графическим интерфейсом и интерфейсом
командной строки, использующие эту библиотеку. Приложения должны
запрашивать входные данные и выполняемую операцию.