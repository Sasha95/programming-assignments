\section{Массивы и коллекции}


\subsection{Одномерные массивы}

\task Дан одномерный массив. Найти количество вхождений в него каждого
элемента.

\task Дан одномерный массив целых чисел. Проверить, образуют ли его
элементы периодическую последовательность и найти её период.

\task Дан одномерный массив вещественных чисел. Заменить нулями
элементы между первым минимальным и последним максимальным элементами.

\task Дан одномерный массив натуральных чисел. Найти количество
простых чисел в нём.

\task Дан одномерный массив. Какой из его элементов встречается чаще
других.

\task Дан одномерный массив целых чисел и некоторое целое число
$p$. Переставить элементы массива так, чтобы сначала были записаны
значения, меньшие $p$, а затем остальные.

\task Дан одномерный массив целых чисел. Удалить из него элементы с
цифрой 5 в записи.

\task Дан одномерный массив вещественных чисел. Определить, сколько в
нём участков максимальной длины со строго возрастающими значениями.

\task Дан одномерный массив целых чисел. Переставить элементы с
нечетными индексами в обратном порядке.

\task Дан одномерный массив целых чисел. Найти индексы первого и
последнего максимального элементов.

\subsection{Многомерные массивы}

\task Дан двумерный массив целых чисел. Заменить нулями элементы,
стоящие в строках и столбцах, в которых есть нулевые элементы.

\task Дан двумерный массив действительных чисел. Заменить каждый элемент
арифметическим средним его смежных соседей по горизонтали, вертикали и
диагонали.

\task Дан двумерный массив размера $M\times N$ действительных
чисел. Нормировать его по формуле
\[
a_{ij}^\textrm{норм} = \frac{a_{ij}-\bar{a}}{\sigma},
\]
где $\bar{a}$ — математическое ожидание, а $\sigma$ — стандартное
отклонение, вычисляемые по формулам
\begin{align*}
  \bar{a} &= \frac{1}{MN}\sum_{i,j}a_{ij},\\
  \sigma  &= \sqrt{\frac{1}{MN}\sum_{i,j}(a_{ij}-\bar{a})^2}. 
\end{align*}

\task Дан двумерный массив действительных чисел. Проверить, есть ли в
нём линейно зависимые строки и столбцы.

\task Заполнить двумерный массив указанного размера последовательными
натуральными числами двигаясь от верхнего левого элемента вправо и
далее по спирали.

Пример заполнения для массива $3\times 3$:
\[
\begin{array}{ccc}
  1 & 2 & 3 \\
  8 & 9 & 4 \\
  7 & 6 & 5
\end{array}
\]


\task Дан двумерный массив целых чисел. Проверить, сколько в нём
квадратов размера $2\times 2$, состоящих только из чётных элементов.

\task Дан квадратный массив действительных чисел. Сформировать массив
с суммами элементов на его диагоналях.

\task Даны две двумерные матрицы. Найти их произведение, если это
возможно.

\task Дан двумерный массив вещественных чисел. Заменить каждый его
элемент арифметическим средним элементов ниже и правее него.

\task Дан двумерный массив. Записать его элементы в одномерный массив
по строкам. При этом нечётные строки обходятся слева направо, а чётные
справа налево.

\subsection{Коллекции}

\task

\task

\task

\task

\task

\task

\task

\task

\task

\task

\subsection{LINQ}

\task Удвоить все элементы коллекции используя LINQ.

\task Найти корень квадратный из каждого элемента коллекции используя
LINQ.

\task В некоторой коллекции заданы углы, найти косинусы этих углов
используя LINQ.

\task Выбрать из коллекции абсолютные значения заданных в ней числовых
значений используя LINQ.

\task В некоторой коллекции заданы радиусы окружностей, найти для
каждой окружности площадь используя LINQ.

\task В некоторой коллекции заданы углы в градусах, найти значения
этих углов в радианах используя LINQ.

\task В некоторой коллекции заданы значения расстояния в метрах
перевести их в дюймы (1 дюйм = 2.54 см) используя LINQ.

\task Возвести каждый элемент коллекции в 5-ую степень используя LINQ.

\task Извлечь из каждого элемента коллекции корень 7-ой степени
используя LINQ.

\task Изменить знаки всех элементов коллекции на противоположные
используя LINQ.


\subsection{Выражения с условием}

\task В некоторой коллекции заданы радиусы окружностей, выбрать только
те, площади окружностей которых не меньше 4используя LINQ.

\task Задана коллекция углов заданных в градусах, выбрать только
острые углы, используя LINQ.

\task В некоторой коллекции заданы углы в радианах, выбрать те,
косинусы которых неотрицательны используя LINQ.

\task Выбрать из коллекции только целые двузначные числа используя
LINQ.

\task Выбрать из коллекции только четные элементы используя LINQ.

\task Выбрать из коллекции только нечетные элементы используя LINQ.

\task Выбрать из коллекции все неотрицательные значения используя
LINQ.

\task Выбрать из коллекции только числа являющиеся полными
квадратами(полные квадраты это числа арифметический корень из которых
есть целое число) используя LINQ.

\task Выбрать из коллекции только числа кратные семи используя LINQ.

\task В коллекции заданы значения углов в радианах, выбрать только
тупые углы, используя LINQ.
