\section{Наследование и полиморфизм}

\subsection{Интерфейсы}

Вы разработчик графических интерфейсов для различных приложений (сами
же приложения пишете не Вы). Поступил заказ на графический интерфейс
для приложения предоставляющего пользователю информацию о состоянии
погоды на данный момент. Вам необходимо будет отображать текущую
температуру и влажность воздуха, скорость и направление ветра,
геомагнитную обстановку. Напишите интерфейс, который должны будут
реализовать разработчики приложения чтобы, вы могли получать
необходимую информацию.

Решение: поскольку информацию придется получать, то было бы неплохо
иметь под рукой соответствующие методы, вызвав которые мы получали бы
интересующие нас данные.

Рассмотрим пример такого интерфейса:

%% public interface IGetinformation
%% {
%%     int GetTempirature(); // получение температуры
%%     int GetWatherPercent();// влажность воздуха
%%     int GetGeomagneticSituation();//геомагнитная обстановка
%%     int GetWindSpeed();//скорость ветра
%%     string GetWindDirection();//направление ветра
%% }


\task Описать интерфейс, который должны будут реализовывать все
коллекции, работающие по принципу списка. Для этого в интерфейсе
должны быть: метод добавления элемента, метод для удаления элемента,
метод возвращающий количество элементов, метод для сортировки
элементов, методы для нахождения максимума и минимума.

\task Описать интерфейс, который должны будут реализовывать все
классы, для экземпляров которых необходимо определить операции
порядка. Для этого в интерфейсе должны быть методы, с помощью которых
можно определить какой из экземпляров больше, меньше, определять их
равенство или неравенство.

\task Вы участвуете в написании игры - симулятора пруда, в котором
живут утки разных пород. Необходимо написать интерфейс который должны
будут реализовать «утки всех пород» для этого в интерфейсе должны быть
методы: перемещение утки вплавь из точки А в точку Б, прием пищи
уткой, кряканье, перелет утки из точки А в точку В, смерть, рождение.

\task Описать интерфейс, привносящий в класс в котором он реализован
следующий функционал: отправка объекта по сети, переименование
объекта, сохранение объекта в файл, вывод объекта в виде строки.

\task Описать интерфейс IVector. В нем должны присутствовать следующие
методы для работы с векторами: сложение двух векторов, вычитание из
одного вектора другого, векторное произведение двух векторов,
скалярное произведение векторов.

\task Описать интерфейс IMatrix. В нем должны присутствовать методы
для выполнения следующих операций над матрицами: нахождение
определителя матрицы, сложение двух матриц, умножение матрицы на
число, умножение двух матриц.

\task Описать интерфейс IStudent. В нем должны присутствовать методы
для выполнения следующих операций со студентом: зачислить, отчислить,
перевести на следующий курс, назначить стипендию (указать размер
стипендии), лишить стипендии.

\task Описать интерфейс ITree. В нем должны присутствовать следующие
методы: сбросить листву, вырасти, зацвести, покрыться листвой, дать
урожай, засохнуть.

\task Описать интерфейс IFigure. В нем должны быть методы для
выполнения с геометрическими плоскими фигурами следующих операций:
найти площадь, найти периметр, вернуть тип(название фигуры).

\task Описать интерфейс I3DFigure. В нем должны быть методы для
выполнения с геометрическими пространствеными фигурами следующих
операций: найти площадь поверхности, найти объем, вернуть тип(название
фигуры).

\subsection{Иерархия классов}

Необходимо описать абстрактный класс Figure, и классы Circle, Тrapeze
и Triangle как производные класса Figure. В классе Figure определить
метод для вычисления площади и переопределить этот метод в каждом из
производных классов.

Решение: Рассмотрим пример программы

%% using System;

%% public abstract class Figure
%% {
%%     public abstract double Square();
%% }

%% public class Circle:Figure
%% {
%%     int rad;
%%     public override double Square()
%%     {
%%         return 2*Math.Pow(Math.PI,2);
%%     }
%% }

%% public class Тrapeze:Figure
%% {
%%     int osnovanie1;
%%     int osnovanie2;
%%     int height;
%%     public override double Square()
%%     {
%%         return height*(osnovanie1+osnovanie2)/2.0;
%%     }
%% }
%% public class Triangle :Figure
%% {
%%     int storona1;
%%     int storona2;
%%     int storona3;
%%     public override double Square()
%%     {
%%         double p = (storona1+storona2+storona3)/2.0;
%%             return Math.Sqrt(p*(p-storona1)*(p-storona2)*(p-storona3));
%%     }
%% }


\task Определить абстрактный класс Car и классы Nissan, Mazda,
Toyota,BMW, как производные класса Car. В классе Car определить метод
возвращающий стоимость автомобиля и его наличие в продаже(2 разных
метода) и переопределить этот метод в каждом из производных классов.

\task Определить абстрактный класс Animal и классы Pig, Cat, Dog, как
производные класса Animal. В классе Animal определить метод
возвращающий информацию о породе животного и переопределить этот метод
в каждом из производных классов.

\task Определить абстрактный класс Telephon и классы Samsung,
Motorola, LG, как производные класса Telephon. В классе Telephon
определить метод возвращающий информацию об объеме оперативной памяти
и размере телефонной книги(количество абонентов) и переопределить эти
методы в каждом из производных классов.

\task Определить абстрактный класс Computer и классы Samsung, Aser,
Asus, как производные класса Computer. В классе Computer определить
метод устанавливающий количество машин данного производителя на складе
и переопределить этот метод в каждом из производных классов.

\task Определить абстрактный класс Game и классы Action, MMORPG,
Strategy, как производные класса Game. В классе Game определить метод
возвращающий название игры и количество дисков на складе и
переопределить эти методы в каждом из производных классов.

\task Определить абстрактный класс OSystem(операционная системма) и
классы Windows, Linux, MacOS, как производные класса OSystem. В классе
OSystem определить метод возвращающий версию операционной системы
(например для Windows может быть возващено Seven или XP и т.п.) и
переопределить этот метод в каждом из производных классов.

\task Определить абстрактный класс Processor и классы Intel, AMD, как
производные класса Processor. В классе Processor определить методы
возвращающие модель процессора и тактовую частоту и переопределить эти
методы в каждом из производных классов.

\task Определить абстрактный класс Lecture и классы Mathimatic,
English, Programing, как производные класса Lecture. В классе Lecture
определить метод возвращающий количество часов отведенных на изучение
данного предмета и переопределить этот метод в каждом из производных
классов.

\task Определить абстрактный класс Book и классы ForCoder,
ForMathematicion, ForIngener, как производные класса Book . В классе
Book определить методы возвращающие название книги и год издания и
переопределить эти методы в каждом из производных классов.

\task Определить абстрактный класс Food и классы Sweet, Fruit, Meat,
как производные класса Food. В классе Food определить методы
возвращающие энергетическую ценность на 100 граммов продукта и срок
хранения в сутках при температуре 18 град. Цельсия и переопределить
эти методы в каждом из производных классов.

\subsection{Обобщённые классы}

Описать класс треугольник(Triangle) с тремя полями соответствующими
сторонам треугольника, методом для вычисления площади и конструктором,
причем тип сторон должен задаваться пользователем при создании
экземпляра класса.

Решение: рассмотрим пример такого класса

%% using System;

%% public class Triangle<T>
%% {
%%     T storona1;
%%     T storona2;
%%     T storona3;
    
%%     public Triangle(T s1,T s2,T s3)
%%     {
%%         storona1 = s1;
%%         storona2 = s2;
%%         storona3 = s3;
%%     }
    
%%     public double Square()
%%     {
%%         double p = (storona1+storona2+storona3)/2.0;
%%             return Math.Sqrt(p*(p-storona1)*(p-storona2)*(p-storona3));
%%     }
        
%% }

В этом примере Т ни что иное как тип сторон который будет указан
пользователем при создании экземпляра класса, например : Triangle<int>
trtiangle = new Triangle<int>(12,11,4);

Так же следует отметить что метод для вычисления площади будет
работать корректно только в случае, если тип Т будет числовым, для
случая когда тип не является числовым следовало бы сгенерировать
соответствующее исключение.

\task Описать класс круг(Circle) с одним полем соответствующим радиусу
окружности, так чтобы пользователь при создании экземпляра класса мог
указать какого типа будет радиус. Определить конструктор и метод для
вычисления площади.

\task Описать класс четырехугольник с четырьмя полями соответствующими
сторонним фигуры, так чтобы при создании экземпляра класса
пользователь сам мог указать какого типа будут стороны. Внутри класса
определить конструктор и метод для вычисления периметра фигуры.

\task Описать класс Apple с двумя полями: масса и цвет - таким образом
чтобы пользователь мог сам указать какого типа будет масса и
цвет. Определить конструктор и метод возвращающий цвет в виде строки
(переопределить метод ToString).

\task Описать класс Car с одним полем Марка Автомобиля таким образом,
чтобы пользователь мог указать при создании экземпляра класса какого
типа будет марка автомобиля. Определить конструктор и метод
устанавливающий марку авто у уже созданного объекта.

\task Описать класс студент с полями текущий год обучения и всего лет
обучаться таким образом чтобы пользователь мог сам указывать типы
полей при создании экземпляра класса. Определить конструктор и метод
показывающий сколько еще лет осталось учиться студенту.

\task Описать класс компьютер с одним полем дата сборки (тип задается
пользователем при создании экземпляра класса). Определить конструктор
и метод вычисляющий возраст машины.

\task Описать класс стол с одним полем модель (тип задается
пользователем при создании экземпляра класса). Определить конструктор
и метод возвращающий значение поля.

\task Описать класс дерево с полями возраст и цвет листвы(тип полей
задается пользователем при создании экземпляра класса). Определить
конструктор и метод возвращающий возраст дерева.

\task Описать класс сфера с одним полем радиус (тип задается
пользователем при создании экземпляра класса). Определить конструктор
и метод возвращающий объем сферы.

\task Описать класс куб с одним полем ребро (тип задается
пользователем при создании экземпляра класса). Определить конструктор
и метод возвращающий площадь поверхности куба.

\subsection{Сравнение экземпляров}

Описать класс Circle и определить в нем операции отношения (<,>,
==,!=).

Решение: Рассмотрим пример программы

%% using System;

%% public class Circle
%% {
%%     public int rad;
%%     public static bool operator < (Circle c1, Circle c2)
%%     {
%%         return c1.rad<c2.rad;
%%     }
    
%%     public static bool operator > (Circle c1, Circle c2)
%%     {
%%         return c1.rad>c2.rad;
%%     }
    
%%     public static bool operator == (Circle c1, Circle c2)
%%     {
%%         return c1.rad==c2.rad;
%%     }
    
%%     public static bool operator != (Circle c1, Circle c2)
%%     {
%%         return c1.rad!=c2.rad;
%%     }
%% }

\task Описать класс Apple(сравнивать по диаметру) и определить в нем
операции отношения (<,>, ==,!=).

\task Описать класс Car(сравнивать по мощности двигателя) и определить
в нем операции отношения (<,>, ==,!=).

\task Описать класс Food(сравнивать по калорийности) и определить в
нем операции отношения (<,>, ==,!=).

\task Описать класс Man(сравнивать по годовому заработку) и определить
в нем операции отношения (<,>, ==,!=).

\task Описать класс Student(сравнивать по году обучения) и определить
в нем операции отношения (<,>, ==,!=).

\task Описать класс Puiple(сравнивать по возрасту) и определить в нем
операции отношения (<,>, ==,!=).

\task Описать класс Tree(сравнивать по высоте) и определить в нем
операции отношения (<,>, ==,!=).

\task Описать класс Processor(сравнивать по тактовой частоте) и
определить в нем операции отношения (<,>, ==,!=).

\task Описать класс Computer(сравнивать по дате выпуска) и определить
в нем операции отношения (<,>, ==,!=).

\task Описать класс Book (сравнивать по количеству страниц в книге) и
определить в нем операции отношения (<,>, ==,!=).
