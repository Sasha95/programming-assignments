\section{Наследование и полиморфизм}

\subsection{Наследование и интерфейсы}

\task

\task

\task 

\task 

\task 

\task 

\task 

\task 

\task 

\task 

\subsection{Обобщённые классы}

\task 

\task 

\task 

\task 

\task 

\task 

\task 

\task 

\task 

\subsection{Сравнение экземпляров}

\task Описать класс Triangle, соответствующий треугольникам.
Определить в нем операции и методы сравнения, сравнивающие
треугольники по площади.

\task Описать класс Time, соответствующий времени суток (часы и
минуты). Определить в нем операции и методы сравнения. Более ранние
моменты времени считать меньшими, чем более поздние.

\task Описать класс Progression, соответствующий геометрическим
прогрессиям. Определить в нём операции и методы сравнения,
сравнивающие прогрессии по значению суммы бесконечного числа их
элементов.

\task Описать класс Box, соответствующий параллелепипедам. Определить
операции и методы сравнения, позволяющие проверить, можно ли вложить
один параллелепипед в другой (в этом случае вложенный считать меньшим,
чем объемлющий). Стенки считать бесконечно тонкими.

\task Описать класс RGBColor, соответствующий цвету в модели RGB. В
этой модели каждый цвет задаётся тремя числами $R,$ $G$ и $B,$
соответствующими интенсивностям красной, зелёной и синей компонент
соответственно. Интенсивности — действительные числа из отрезка
$[0; 1]$.

Определить в классе операции и методы сравнения, сравнивающие цвета по
яркости. Яркость вычисляется по формуле:
\[
Y = 0{,}299 R + 0{,}587 G + 0{,}114 B.
\]

\task Описать класс Deposit, соответствующий банковскому вкладу. Вклад
определятся тремя величинами: начальной суммой, сроком (в годах) и
процентной ставкой. Каждый год сумма на счету увеличивается на
величину процентной ставки.

Определить в классе операции и методы сравнения, сравнивающие вклады
по величине чистой прибыли за срок действия вклада.

\task Описать класс Interval, соответствующий отрезкам числовой
прямой. Определить в классе операции и методы сравнения, сравнивающие
отрезки следующим образом. Из двух пересекающихся отрезков больше тот,
у которого доля общей части больше. Непересекающиеся отрезки считаются
разными.

\task Описать класс Rate, соответствующий пакетам услуг связи. Каждый
пакет услуг определяется стоимостью, количеством минут в пакете и
количеством бесплатных минут.

Определить в классе операции и методы сравнения, сравнивающие пакеты
по экономичности. Более экономичным является тот пакет услуг, в
котором стоимость одной минуты ниже.

\task Описать класс Point, соответствующий точкам в пространстве.
Определить в классе операции и методы сравнения. Считать меньшей ту
точку, которая ближе к началу координат.

\task Описать класс Line отрезков на плоскости. Определить в нем
операции и методы сравнения, сравнивающие отрезки по длине.
