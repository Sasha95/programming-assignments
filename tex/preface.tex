\SectionS{Введение}

Единственный способ научиться программировать — это
программировать. Только решая задачи, сложные или простые, можно
получить необходимые навыки и опыт.

Существует большое количество сборников задач по программированию, но
лишь небольшая их часть удовлетворяет требованиям современных
университетских курсов. Настоящий сборник представляет собой попытку
восполнить этот пробел. Была поставлена цель подобрать или составить
такие задачи, чтобы они охватывали весь курс программирования, не
только конструкции управления процессом управления. В частности, в
сборнике есть задачи на применение современной технологии обработки
данных LINQ, написание регулярных выражений, проектирование классов.

%Будет одна часть или две?

Первая часть сборника содержит более 300 задач. Предполагается, что
задачи будут решаться на языке программирования C\#, но в большинстве
случаев выбор языка программирования непринципиален.

У некоторых задач намеренно оставлена нечёткая формулировка
условий. Во-первых, это приближение к реальной работе, когда чёткие
постановки задач — редкость. Во-вторых, это развивает навыки анализа
задачи, умение отделить главное от второстепенного. Кроме того,
студентам не навязываются искусственные ограничения, такие, как запрет
на использование каких-либо конструкций языка при решении.

Некоторые задачи поучительны: рассказывают об интересных фактах,
заставляют вспомнить результаты из смежных областей знаний, таких как
математика и физика.

Почти каждую из задач можно решить «в лоб». Но во многих случаях (и
это тоже сделано намеренно), первое очевидное решение не будет самым
эффективным. Немного подумав, можно сократить его всего до нескольких
команд.

Номер каждой задачи состоит из трёх частей: номеров темы и раздела, а
также варианта. Всего вариантов 10 — от 0 до 9. Такой подход позволяет
легко распределить задачи между студентами. Например, вариант можно
назначить в соответствии с последней цифрой номера зачётной книжки или
номера в списке группы.

%Рассказать про источники задач (с номерами). Часть ис сборников, а часть
%оригинальные.

%Благодарности

%Рекомендуемая литература

Исходные тексты последней версии сборника находятся на веб-странице
пособия —
\url{https://github.com/velikodniy/programming-tasks}. О найденных
опечатках или неточностях можно сообщить по адресу электронной почты
\href{mailto:vadim@veikodniy.name}{vadim@velikodniy.name}.
