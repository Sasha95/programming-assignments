\section{Обработка текстовых данных}

\subsection{Простые преобразования}

\task Заменить в тексте все символы «*» на символ «.», и в конец
текста добавить предложение вида «замены были в позициях
  N1,N2,N3,…,Nk», где Ni индекс замененного символа в тексте.

\task Дано предложение. Заменить в нем буквосочетание да на нет.

\task Дано предложение заменить в нем все буквы а на о.

\task Дано предложение. Поменять местами его вторую и пятую букву.

\task Дано предложение. Заменить все символы, стоящие на четных местах
символом «*».

\task Дано слово. Поменять в нем первую букву со второй, третью с
четвертой и т.д.

\task Дано слово из 12-ти букв. Переставить в обратном порядке буквы,
расположенные между второй и десятой буквами.

\task Дано предложение. Заменить в нем все вхождения буквосочетаний
бип на рог.

\task Дано предложение. Все символы, расположенные на третьем шестом
девятом и т.д. местах заменить на букву а.

\task Дано предложение. Заменить в нем все вхождения буквы а на
буквосочетание ух.


\subsection{Стандартные методы}

\task Массив чисел задан в виде строки. Найти сумму всех четных
элементов.

\task Массив чисел задан в виде строки. Уменьшить каждый элемент на 3.

\task Задан массив чисел. Записать его в виде строки, а в качестве
разделителя использовать символ «;».

\task Дан текст вида «d1+d2+d3+…+dn», где di >1 вычислить сумму,
записанную в строке.

\task Дан текст вида «d1+d2-d3+d4-d5+…+dn» вычислить сумму, записанную
в строке.

\task Массив чисел задан в виде строки. Вычислить среднее
арифметическое.

\task Дано предложение из десяти слов, заполнить ими массив на десять
элементов.

\task Выяснить, сколько слов содержится в строке.

\task Задано сложное предложение выяснить сколько простых предложений
в нем содержится(предложение сложное).

\task Дан текст вида «d1*d2*d3*…*dn», где di >1 вычислить
произведение, записанное в строке

\subsection{Регулярные выражения}

\task Заменить везде в тексте имя «Вася» на «Коля».

\task Везде, где в тексте встречаются суммы, заменить их на результат
вычисления(например подстроку вида «2+4» заменить на 6).

\task Посчитать, сколько чисел в тексте.

\task Выбрать из текста все адреса электронной почты и поместить их в
список.

\task Задано название файла с полным путем к нему, выделить пути
только название файла с расширением(если расширение имеется).

\task Подсчитать сколько раз в тексте встречается слово JavaScript.

\task Задан текст. В каждом предложении первую букву сделать заглавной

\task Задан текст. Заменить в нем Java на C\#.

\task Задан текст. Заменить в нем все разности результатом вычисления
(например «4-3» заменить на 1)

\task Удалить из текста слово «начало» если оно находится в конце
предложения

