\section{Обработка текстовых данных}

\subsection{Стандартные методы}

\task Дан фрагмент текста, запрашиваемый у пользователя. Написать
программу, находящую наибольшее количество цифр, идущих в нём подряд.

\task Дан фрагмент текста, запрашиваемый у пользователя. Написать
программу, определяющую, на каких позициях в нём встречаются пробелы.

\task Дан фрагмент текста, запрашиваемый у пользователя. Написать
программу, заменяющую в нём все прописные буквы на строчные после
первого вхождения символа «*».

\task Дан фрагмент текста, запрашиваемый у пользователя. Написать
программу, выводящую частоту каждого из его символов.

\task Дан фрагмент текста, запрашиваемый у пользователя. Написать
программу, меняющую регистр всех букв на противоположный.

\task Дана фрагмент текста, запрашиваемый у пользователя. Фрагмент
содержащит слова, разделённые пробелами (одним или
несколькими). Написать программу, выводящую те же слова через один
пробел в обратном порядке.

\task Дана фрагмент текста, запрашиваемый у пользователя. Фрагмент
содержит запись арифметического выражения. Написать программу,
проверяющую, правильно ли в нём расставлены круглые скобки.

\task Дан фрагмент текста, запрашиваемый у пользователя. Написать
программу, заменяющую в нём гласные русские буквы на символ «*».

\task Дан фрагмент текста, запрашиваемый у пользователя. Написать
программу, проверяющую, правильно ли в нём записаны буквосочетания
«жи», «ши», «ча», «ща».

\task Дан фрагмент текста, запрашиваемый у пользователя. Написать
программу, проверяющую, является ли он палиндромом, то есть читается
ли одинаково слева направо и справа налево. При проверке регистр букв,
знаки препинания и пробелы не учитываются. Например, строка «Аргентина
манит негра.» — палиндром.

\subsection{Регулярные выражения}

\task Назовём идентификатором последовательность латинских букв, цифр
и знаков подчёркивания, начинающуюся не с цифры. Дан фрагмент текста,
запрашиваемый у пользователя. Написать программу, определяющую,
сколько в нём различных идентификаторов (регистр символов не
учитывать).

\task Дан фрагмент текста, запрашиваемый у пользователя. Написать
программу, заменяющую встречающиеся в тексте суммы пары натуральных чисел
на результат суммирования. Например, строку вида «Сумма равна 12+4.»
требуется заменить на «Сумма равна 16.».

\task Дан фрагмент текста, запрашиваемый у пользователя. Написать
программу, находящую сумму целых чисел (возможно, со знаком),
встречающихся в нём.

\task Дан фрагмент текста, запрашиваемый у пользователя. Написать
программу, выводящую на экран список адресов электронной почты без
повторов, встречающихся в нём. Считать, что адрес электронной почты
имеет вид «пользователь@сервер», где имя пользователя и название
сервера могутт состоять из латинских букв, цифр, дефисов и точек.
Кроме того адрес не может заканчиваться на точку.

\task Дана фрагмент текста, запрашиваемый у пользователя. Фрагмент
содержит список имён и фамилий. В качестве разделителя используется
точка с запятой — «Имя Фамилия; Имя Фамилия; …». Написать программу,
преобразующую список к формату «Фамилия, Имя; Фамилия, Имя; …».

\task Дан фрагмент текста, запрашиваемый у пользователя. Написать
программу, проверяющую, является ли он записью вещественного числа.

\task Дан фрагмент текста на русском языке, запрашиваемый у
пользователя. Написать программу, переводящую первую букву каждого
предложения в верхний регистр.

\task Дан фрагмент текста на русском языке, запрашиваемый у пользователя.
Написать программу, определяющую, сколько в нём слов, букв и символов.
Словом считается последовательность букв, содержащая не более, чем один
дефис. Слово не может начинаться с дефиса изи заканчиваться на него.

\task Дан фрагмент текста, запрашиваемый у пользователя. Написать
программу, проверяющую, встречаются ли в нём идущие подряд одинаковые
слова.

\task Дан фрагмент текста, запрашиваемый у пользователя. Написать
программу, выводящую на экран список номеров телефонов без повторов,
встречающихся в нём. Номера телефонов в тексте записаны в формате
«(код)номер» и могут содержать пробелы или знаки «-» (выводить нужно
без них).
