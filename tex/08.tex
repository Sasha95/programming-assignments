\section{Обработка текстовых данных}

\subsection{Стандартные методы}

\task Дана строка. Написать программу, находящую наибольшее количество
цифр, идущих в ней подряд.

\task Дана строка. Написать программу, определяющую, на каких позициях
в ней встречаются пробелы.

\task Дана строка. Написать программу, заменяющую в ней все прописные
буквы на строчные после первого вхождения символа «*».

\task Дана строка. Написать программу, выводящую частоту каждого из её
символов.

\task Дана строка. Написать программу, меняющую регистр всех букв на
противоположный.

\task Дана строка, содержащая слова, разделённые пробелами (одним или
несколькими). Написать программу, выводящую те же слова через один
пробел в обратном порядке.

\task Дана строка, содержащая запись арифметического
выражения. Написать программу, проверяющую, правильно ли в нём
расставлены круглые скобки.

\task Дана строка. Написать программу, заменяющую в ней гласные
русские буквы на символ «*».

\task Дан фрагмент текста. Написать программу, проверяющую, правильно
ли в нём записаны буквосочетани «жи», «ши», «ча», «ща».

\task Дан фрагмент текста. Написать программу, проверяющую, является
ли он палиндромом, то есть читается ли одинаково слева направо и
справа налево. При проверке регистр букв, знаки препинания и пробелы
не учитываются. Например, строка «Аргентина манит негра.» — палиндром.

\subsection{Регулярные выражения}


\task Назовём идентификатором последовательность латинских букв, цифр и знаков подчёркивания, начинающуюся не с цифры. Дана строка. Написать программу, определяющую, сколько в ней различных идентификаторов (регистр символов не учитывать).

*\task Везде, где в тексте встречаются суммы, заменить их на результат
вычисления(например подстроку вида «2+4» заменить на 6).

\task Дана строка. Написать программу, находящую сумму целых чисел (возможно, со знаком), встречающихся в ней.

\task Дан фрагмент текста. Написать программу, выводящую на экран список адресов электронной почты без повторов, встречающихся в нём. Считать, что адрес электронной почты имеет вид «пользователь@сервер», где имя пользователя может состоять из латинских букв, цифр, дефисов и точек. Кроме того адрес не может заканчиваться на точку.

\task Дана строка. Написать прогармму

\task Дана строка. Написать программу, проверяющую, является ли она записью вещественного числа.

*\task Задан текст. В каждом предложении первую букву сделать заглавной

\task Дан фрагмент текста. Написать программу, определяющую, сколько в нём слов.

*\task Задан текст. Заменить в нем все разности результатом вычисления
(например «4-3» заменить на 1)

\task Дан фрагмент текста. Написать программу, выводящую на экран список номеров телефонов без повторов, встречающихся в нём. Номера телефонов в тексте записаны в формате «(код)номер» и могут содержать пробелы или знаки «-» (выводить нужно без них).

