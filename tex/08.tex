\section{Обработка текстовых данных}

\subsection{Простые преобразования}

Посчитать количество букв “ё” в тексте и в конец текста с новой строки
добавить “В предложении n букв ё.” , где n количество букв ё.

Решение: рассмотрим два способа решения этой задачи: с помощью
StringBuilder и с помощью конкотенации строк.

В цикле пройдем по строке и подсчитаем количество букв ё и в
зависимости от их количества добавим или не добавим соответствующую
строку.

Текст программы:

%% using System;
%% using System.Collections.Generic;
%% using System.Linq;
%% using System.Text;

%% namespace test
%% {
%%     class Program{
%%         public static void Main(string[] args)
%%         {
            
                
%%                 //с использованием StringBuilder - здесь мы можем добавлять и удалять содержимое из строки
%%         StringBuilder textB = new StringBuilder();
%%         textB.Append(@''В лесу живет ёжик. У ёжика есть иголки'');
%%                 int k=0;
%%             for (int i = 0; i < textB.Length; i++) {
%%                     if(textB[i]=='ё')
%%                         k++;
%%                 }
%%                 textB.AppendLine(String.Format(``В предложении {0} букв ё.'',k));
%%                 Console.WriteLine(textB);
%%                 Console.ReadKey(true);
                
%%                 //без StringBuilder - здесь чтобы добавить в строку содержимое нужно создавать новую строку
%%                 string text = @'' В лесу живет ёжик. У ёжика есть иголки ``;
%%             k=0;
%%             for (int i = 0; i < text.Length; i++) {
%%                     if(text[i]=='ё')
%%                         k++;
%%                 }
%%                 var newTxt = text + String.Format(``В предложении {0} букв ё.'',k);
%%                 Console.WriteLine(newTxt);
%%                 Console.ReadKey(true);
%%             }
            
%%         }
%%     }

\task Заменить в тексте все символы «*» на символ «.», и в конец
текста добавить предложение вида «замены были в позициях
  N1,N2,N3,…,Nk», где Ni индекс замененного символа в тексте.

\task Дано предложение. Заменить в нем буквосочетание да на нет.

\task Дано предложение заменить в нем все буквы а на о.

\task Дано предложение. Поменять местами его вторую и пятую букву.

\task Дано предложение. Заменить все символы, стоящие на четных местах
символом «*».

\task Дано слово. Поменять в нем первую букву со второй, третью с
четвертой и т.д.

\task Дано слово из 12-ти букв. Переставить в обратном порядке буквы,
расположенные между второй и десятой буквами.

\task Дано предложение. Заменить в нем все вхождения буквосочетаний
бип на рог.

\task Дано предложение. Все символы, расположенные на третьем шестом
девятом и т.д. местах заменить на букву а.

\task Дано предложение. Заменить в нем все вхождения буквы а на
буквосочетание ух.

\subsection{Стандартные методы}

Массив чисел задан в виде строки a1 a2 a3 a4 a5 … an . Необходимо
найти максимум и минимум.

Решение: С помощью LINQ и стандартных методов классов Array(Max,Min) и
String (Split) найдем необходимые значения.

Пример программы:

%% using System;
%% using System.Collections.Generic;
%% using System.Linq;
%% using System.Text;

%% namespace test
%% {
%%     class Program
%%     {
%%         public static void Main(string[] args)
%%         {
%%             var DATA = @''1 2 3 4 5 6 7 8 9 0 5000 0,23 -400'';
%%             var numerics= from x in DATA.Split(' ')/*<- разбиваем строку на масив строк по символу пробел и перебираем его элементы*/ 
%%                 select double.Parse(x)/* <- преобразуем строку в число типа double*/;
%%             Console.WriteLine(``Maximum: {0}\nMinimum: {1}'',numerics.Max(),numerics.Min());
%%             Console.ReadKey(true);
%%         }
%%     }
%% }


\task Массив чисел задан в виде строки. Найти сумму всех четных
элементов.

\task Массив чисел задан в виде строки. Уменьшить каждый элемент на 3.

\task Задан массив чисел. Записать его в виде строки, а в качестве
разделителя использовать символ «;».

\task Дан текст вида «d1+d2+d3+…+dn», где di >1 вычислить сумму,
записанную в строке.

\task Дан текст вида «d1+d2-d3+d4-d5+…+dn» вычислить сумму, записанную
в строке.

\task Массив чисел задан в виде строки. Вычислить среднее
арифметическое.

\task Дано предложение из десяти слов, заполнить ими массив на десять
элементов.

\task Выяснить, сколько слов содержится в строке.

\task Задано сложное предложение выяснить сколько простых предложений
в нем содержится(предложение сложное).

\task Дан текст вида «d1*d2*d3*…*dn», где di >1 вычислить
произведение, записанное в строке

\subsection{Регулярные выражения}

По всем правилам запятая ставиться сразу после слова и после неё
обязательно должен следовать пробел, например: «не черное, а
  белое». Задан текст, в котором запятые расставлены без соблюдения
этого правила. Написать программу расставляющую запятые правильно.

Решение: к сожалению, в .NET реализовано не все множество возможностей
регулярных выражений языка PERL 5, но даже того что есть будет
достаточно. Для поиска некорректных расстановок запятых воспользуемся
шаблоном @“\\s*,\\s*” , где \\s – соответствует пробельному символу, а *
указывает что он может повторяться 0 и более раз.

Пример программы:

%% using System;
%% using System.Collections.Generic;
%% using System.Linq;
%% using System.Text.RegularExpressions;

%% namespace test
%% {
%%     class Program
%%     {
%%         public static void Main(string[] args)
%%         {
%%             var DATA = @''Mozilla exists to promote openness,    innovation and opportunity on the Internet.
%%  We’ve been doing it for 15 years and the following 15 facts offer
%%  a look at who we are,including some of our biggest achievements and milestones.
%%  But in some ways we’re just getting started , exploring new technology   , entering
%%  new areas and reaching new users every day. ``;
%%             var rightText = Regex.Replace(DATA,@''\s*,\s*'','', ``,RegexOptions.Multiline);
%%             Console.WriteLine(rightText);
%%             Console.ReadKey(true);
%%         }
%%     }
%% }

\task Заменить везде в тексте имя «Вася» на «Коля».

\task Везде, где в тексте встречаются суммы, заменить их на результат
вычисления(например подстроку вида «2+4» заменить на 6).

\task Посчитать, сколько чисел в тексте.

\task Выбрать из текста все адреса электронной почты и поместить их в
список.

\task Задано название файла с полным путем к нему, выделить пути
только название файла с расширением(если расширение имеется).

\task Подсчитать сколько раз в тексте встречается слово JavaScript.

\task Задан текст. В каждом предложении первую букву сделать заглавной

\task Задан текст. Заменить в нем Java на C\#.

\task Задан текст. Заменить в нем все разности результатом вычисления
(например «4-3» заменить на 1)

\task Удалить из текста слово «начало» если оно находится в конце
предложения

