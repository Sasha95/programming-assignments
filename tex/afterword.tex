\SectionS{Заключение}

Чтобы научиться программировать — нужно программировать. Наверняка
читатель уже понял эту истину во время решения задач из сборника. Но
важно помнить, что обучение (и самообучение) — это непрерывный
процесс. Поэтому не стоит останавливаться на достигнутом. Есть и
другие задачники и учебники.

Для человека, который уже имеет определённые навыки в области
программирования, наверное, имеет смысл решать не отдельные задачи, а
попробовать реализовать небольшой проект. Это позволит приобрести
знания и навыки в областях, которые этот сборник, к сожалению, обошёл
стороной: в разработке сложных систем, в проектировании архитектуры, в
одновременном сочетании различных технологий.

Но в любом случае важность и полезность практики переоценить
невозможно.



