\SectionS{Обзор литературы}

Для успешного решения задач нужно, безусловно, владеть теоретическими
знаниями в области программирования. Этот задачник ориентирован на
язык программирования C\#, поэтому ниже приводится обзор наиболее
удачных, по мнению автора книг, позволяющих выучить этот
язык. Впрочем, задачи необязательно решать именно на C\#. По большей
части (за исключением лишь нескольких тем) они составлены так, чтобы
их можно было решить практически на любом языке.

Кроме книг по C\# ниже также перечислены полезные книги общего
характера о программировании, алгоритмах и технологиях разработки
программного обеспечения.

Разумеется, хоть все эти книги и заслуживают внимания, не стоит читать
их все одновременно. Лучше просмотреть их, а потом выбрать для себя
пару книг, которые будут основными. И уже их внимательно прочитать и
выполнить упражнения.

Скажем, если вы плохо знакомы с программированием, то в качестве
первой книги о языке C\#, наверное, стоит взять самоучитель Шилдта
\cite{schildt}. Если же у вас уже есть опыт программирования и для вас
нет необходимости разбираться в том, что такое переменная или цикл, то
изучить язык можно и по справочнику Албахари \cite{albahari}.

В списке есть книги, которые стоит прочитать всем, вне зависимости от
уровня и опыта. По мнению автора, это руководство по написанию
качественных программ Макконнелла \cite{mcconnell} и учебник по
разработке и анализу алгоритмов Седжвика \cite{sedgewick}
(используемый, кстати, в Принстонском университете).

Тех, кто только начал изучать программирование, возможно заинтересует
вводный курс программирования Гарвардского университета
\cite{cs50}. Он в живой и доступной форме знакомит с основными
понятиями, универсальными для разработки приложений для компьютеров.

\begingroup
\renewcommand{\section}[2]{}%
\begin{thebibliography}{99}

  \subsection*{Язык C\#}

\bibitem{albahari}
  Албахари Д., Албахари Б. C\# 5.0. Справочник. Полное описание языка. — М.: Вильямс, 2014.

  Несмотря на то, что это справочник, эту книгу можно использовать для
  изучения различных возможностей языка. Книга подробная, написана
  простым языком и отлично подойдёт тем, кто уже знаком с основами
  программирования.

\bibitem{schildt}
  Шилдт Г. C\# 4.0: Полное руководство. — М.: Вильямс, 2011.

  Книга ориентирована на начинающих программистов, объяснение ведётся
  «с нуля».

\bibitem{troelsen}
  Троелсен Э. Язык программирования C\# 5.0 и платформа .NET 4.5. — 6-е изд. — М.: Вильямс, 2013.

  Читается немногим сложнее Шилдта, но зато рассматривается более
  широкий круг тем.
  
\bibitem{richter}
  Рихтер Дж. CLR via C\#. Программирование на платформе Microsoft .NET Framework 4.0 на языке C\#. 4-е изд. — СПб.: Питер. 2012.

  Эта книга для тех, кто уже знаком с программированием под платформу
  .NET и хочет узнать детали. Книга очень хорошая, но сложновата для
  начинающих. К тому же она больше посвящена платформе .NET, чем C\#.

\bibitem{skeet}
  Скит Дж. C\#. Программирование для профессионалов. — М.: Вильямс, 2011.

  Отличная книга, но, как и предыдущая, ориентированная больше на тех,
  кто уже знаком с C\# на начальном уровне.

\bibitem{metanit}
  Полное руководство по языку программирования С\# 6.0 и платформе .NET 4.6 / http://metanit.com/sharp/tutorial/

  Написанное простым языком краткое руководство по основам языка C\#,
  рассчитанное на начинающих.

\bibitem{professorweb}
  C\# 5.0 и платформа .NET 4.5 / http://professorweb.ru/

  Ещё один, уже более подробный, онлайн-учебник. В нём рассказывается
  не только о языке C\#, но и о применении его для разработки
  веб-приложений.

  \subsection*{Программирование}

\bibitem{mcconnell}
  Макконнелл С. Совершенный код. — СПб.: Питер, 2007.

  Великолепная книга о том, как надо программировать. Она не посвящена
  какому-то языку программирования или технологии, а скорее
  представляет собо большой сборник советов и рекомендация, как
  организовать свою работу, как правильно писать программный код и так
  далее.

\bibitem{frield}
  Фридл Дж. Регулярные выражения. — СПб.: Символ-Плюс, 2000.

  Всеобъемлющее руководство по регулярным выражениям и их применению.

\bibitem{booch}
  Буч~Г., Максимчук~Р.~А., Энгл~М.~У., Янг~Б.~Дж., Коналлен~Дж., Хьюстон~К.~А. Объектно-ориентированный анализ и проектирование с примерами приложений (UML 2). — 3е издание. — М.: Вильямс, 2010.

  Эта книга поможет лучше понять концепцию объектно-ориентированного
  программирования и познакомиться с языком UML.

\bibitem{abelson}
  Абельсон Х., Сассман Дж. Дж. Структура и интерпретация компьютерных программ. — М.: Добросвет, 2006.

  Один из лучших учебников программирования, использующий, однако,
  язык Scheme для иллюстрации объяснения.

\bibitem{cs50}
  CS50 Основы программирования / http://javarush.ru/cs50.html

  Перевод на русский язык видеозаписи курса основ программирования
  Гарвардского университета.  Курс считается одним из лучших, и в то
  же время он достаточно простой для самостоятельного изучения.

  \subsection*{Алгоритмы}
  
\bibitem{cormen}
  Кормен Т. Х., Лейзерсон Ч. И., Ривест Р. Л., Штайн К. Алгоритмы: построение и анализ, 3-е изд. — М.: Вильямс, 2013.
  
  Один из лучших учебников по алгоритмам. Может использоваться и как
  справочник. В третьем издании была добавлена глава о параллельных
  алгоритмах.
  
\bibitem{sedgewick}
  Седжвик Р., Уэйн К. Алгоритмы на Java. — М.: Вильямс, 2013.

  В отличие от предыдущей книги, эта ориентирована на изучение
  практических аспектов разработки алгоритмов. Отличительной
  особенностью является большое количество упраженений, позволяющих
  закрепить материал.
  
\bibitem{wirth}
  Вирт Н. Алгоритмы и структуры данных. — М.: Мир, 1989. — 360 с.

  Книга от создателя языка Pascal, посвящённая разработке и анализу
  алгоритмов. Менее подробна, чем предыдущая, но написана более
  простым языком и ориентирована на более широкий круг читателей.

\subsection*{Технологии}

\bibitem{git-parable}
  Басня о Git / http://hades.github.io/2009/05/the-git-parable-ru/
  
  Рассказ о том, зачем нужен Git и распределённые системы контроля версий.

\bibitem{git-howto}
  Git How To / https://githowto.com/ru

  Интерактивный обучающий курс по системе контроля версий Git, которую
  настоятельно рекомендуется (а в дальнейшем обязательно требуется)
  использовать для хранения исходных текстов программ.
\end{thebibliography}
\endgroup
