\section{Алгоритмы с ветвлением}

\subsection{Условный оператор}

\task Три точки на плоскости заданы своими координатами. Написать
программу, определяющую, лежат ли они на одной прямой.

\task Дан шестизначный номер билета. Написать программу, проверяющую,
является ли билет «счастливым». (Билет будем считать «счастливым»,
если сумма первых трёх цифр равна сумме последних трёх цифр.)

\task Стоимость минуты разговора по телефону — $p~\textup{руб.}$ Если
продолжительность разговора превышает 5 минут, то на оставшуюся часть
времени действует скидка, равная 20\,\%. Написать программу,
определяющую стоимость разговора заданной продолжительности $t$.

\task Написать программу, проверяющую, равно ли утроенное произведение
цифр заданного удвоенного числа ему самому. (Например, число 15
удовлетворяет этому условию).

\task Написать программу, определяющую, является ли заданное
четырёхзначное число палиндромом. (Число-палиндром — это число, запись
которого слева направо совпадает с записью справа налево.)

\task Даны радиус окружности и катеты прямоугольного
треугольника. Написать программу определяющую, можно ли вписать
треугольник в окружность.

\task Написать программу, вычисляющую для заданного $n$ значение функции
\[
f(n) =
\begin{cases}
  n/2,  &\textup{если }n\textup{ чётное,} \\
  3n+1, &\textup{если }n\textup{ нечётное.}
\end{cases}
\]

\task Дан прямоугольник размерами $w\times h.$ Написать программу,
определяющую, можно ли полностью покрыть его $n$ плитками размера
$a\times a.$

\task Снаряд выпущен под углом $\alpha$ к горизонту с начальной
скоростью $v$. Написать программу, проверяющую, попадёт ли он в цель
высотой $h$, находящуюся на расстоянии $L$ от пушки. (Ускорение
свободного падения $g\approx 9{,}81~\frac{\textup{м}}{\textup{с}^2}$,
сопротивлением воздуха пренебречь.)

\task Дано целое число $k$ ($1 \leqslant k \leqslant 365$). Написать
программу, определяющую, придётся ли $k$-й день года на воскресенье,
если 1 января — понедельник.

\subsection{Составные условия} 

\task Даны две стороны треугольника и угол между ними. Составить
программу, определяющую, является ли треугольник равносторонним.

\task Написать программу, определяющую, является ли указанный год
високосным. (Год не является високосным, если его номер не кратен 4,
либо кратен 100, но при этом не кратен 400.)

\task Даны два целых числа. Написать программу, определяющую, является
ли наименьшее из них чётным.

\task Написать программу, определяющую, расположены ли цифры заданного
четырёхзначного числа в нём по возрастанию слева направо.

\task Написать программу, определяющую, можно ли из отрезков с длинами
$a$, $b$ и $c$ составить треугольник.

\task Квадрат, со сторонами, параллельными осям координат, задан
координатами левого верхнего и нижнего правого углов. Написать
программу, проверяющую, находится ли точка $(x, y)$ внутри квадрата.

\task Дано трёхзначное число. Написать программу, определяющую,
содержит ли оно нечётные цифры.

\task Даны два отрезка числовой прямой. Написать программу,
определяющую, пересекаются ли они.

\task На плоскости даны точки $A(x_A, y_A),$ $B(x_B, y_B)$ и $C(x_C,
y_C)$. Написать, программу, определяющую расстояние от точки $(x, y)$
до ближайшей из указанных.

\task Написать программу, проверяющую, можно ли вставить стержень с
сечением в виде прямоугольника с размерами $w×h$ в прямоугольное
отверстие с размерами $W×H.$

\subsection{Несколько условий или оператор выбора}

\task Написать программы, выводящую три заданных числа в порядке
возрастания.

\task В старояпонском календаре был принят 12-летний цикл, в котором
года носили названия животных: крыса, корова, тигр, заяц, дракон,
змея, лошадь, овца, обезьяна, курица, собака и свинья. Например, 1988
год был годом дракона. Написать программу, определяющую животное,
соответствующее указанному году.

\task Написать программу, решающую квадратное уравнение по заданным
коэффициентам.

\task Написать программу, находящую медиану трёх заданных
чисел. (Медиана — это число, расположенное посередине упорядоченного
списка.)

\task Даны три стороны треугольника. Написать программу, определяющую,
является ли он прямоугольным, равнобедренным или
равносторонним. (Треугольник может относиться к нескольким классам
одновременно.)

\task Дано целое число $k$ ($1 \leqslant k \leqslant 365$). Написать
программу, определяющую день недели $k$-го дня года, если 1 января —
понедельник.

\task Написать программу, определяющую, сколько различных трёхзначных
чисел можно составить из цифр заданного трёхзначного числа.

\task Написать программу, определяющую количество решений системы
линейных уравнений с двумя неизвестными.

\task Две окружности заданы координатами центров и радиусами. Написать
программу, определяющую количество точек их пересечения.

\task Некоторый день задан двумя числами: $d$ (день), $m$
(месяц). Написать программу, проверяющую, возможна ли такая дата, а
также указываюшую, где ошибка: в номере дня или
месяца. Предполагается, что год невисокосный. (Например, невозможны
даты $(d=34, m=11),$ $(d=-5, m=7)$ и $(d=3, m=14)$.)
