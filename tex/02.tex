\section{Алгоритмы с ветвлением}

\subsection{Условный оператоп}

\task Написать программу, выводящую на экран наименьшую и наибольшую из стоимостей 2-х товаров и их сумму, если она больше стоимости первого товара не более чем в три раза.

\task Даны две пары чисел. Написать программу, вычисляющую сумму наименьшего числа из первой пары с наибольшим числом из второй.

\task Даны 2 числа. Написать программу,которая увеличивает первое число в 10 раз, если второе число нечётное.

\task Дано трёхзначное число a. Написать программу, проверяющую, сожержит ли m цифру n.

\task Написать программу, пределяющую будет ли данное четырёхзначное число палиндромом.

\task Даны радиус окружности и катеты прямоугольного треугольника.Напиать программу определяющую наибольшую из площадей.

\task Написать программу, вычисляющую значение функции $y= x^6-5*x+3, x<0;
1-lnx-5*x, x>=0.$

\task Даны скорости двух велосипедистов и расстояние до соседнего посёлка. Определить кто из них первым прибудет в этот посёлок.

\task Известны расстояния от одной точки до двух разных объектов. До первого n км, до второго k футов. Определить какое из расстояний больше.(1 фут=0,45 м)

\task Даны катеты двух треугольников. Написать программу, находящую наименьшую из площадей. В случае если наименьшая площадь первого треугольника вывести так же сумму площадей треугольника.

\subsection{Сложные условия}

\task Написать программу, определяющую является ли данное число трёхзначным и чётным.

\task Даны 2 числа, определить является ли наименьшее из них чётным.

\task Написать программу определяющую будут ли цифры данного четырёхзначного числа, записанные в том же порядке, что и в числе, образовавать возрастающую последовательность.

\task Составить программу, определяющую является ли треугольник равносторонним.

\task Даны числа a и b. Написать программу, которая определяет делятся ли эти числа на 3, 5, 7 одновременно.

\task Написать программу, определяющую являются ли a, b и c сторонами треугольника.

\task Дано пятизначное число, определить содержет ли оно цифры 3 и 4 или 1 и 7.

\task Написать программу, определяющую является ли g остатком от деления n на k либо k на n.

\task Написать программу, определяющую, лежит ли точка с абсциссой x в одном из интервалов [-10,5], [10,15].

\task Составить программу, которая определяет делится ли сумма цифр данного четырёхзначного число на n и при этом является ли она двузначной.

\subsection{Вложенные условные операторы}

\subsection{Оператор выбора}
