\section{Функции}

\subsection{Процедуры}

\task 

\task

\task

\task

\task

\task

\task

\task

\task

\task 

\subsection{Функции}

\task Даны $n$ натуральных чисел. Написать программу, находящую их
общий делитель. (Определить функцию для расчёта наибольшего общего
делителя двух чисел.)

\task Даны основания и высоты двух равнобедренных трапеций. Написать
программу, находящую сумму их периметров. (Определить функцию для
расчёта периметра равнобедренной трапеции по её основанию и высоте.)

\task Даны три квадратных уравнения:
\begin{align*}
a x^2 + b x + c &= 0,\\
b x^2 + a x + c &= 0,\\
c x^2 + a x + b &= 0,
\end{align*}
где $a,b,c\neq 0.$
Написать программу, определяющую, сколько из них имеют действительные
корни. (Определить функцию, позволяющую распознавать наличие
вещественных корней в квадратном уравнении.)

\task Написать программу для вычисления биномиального коэффициента
\[
C_n^k = \frac{n!}{k!(n-k)!}
\]
для заданных $n$ и $k$. (Определить функцию для вычисления факториала
числа.)

\task Два простых числа называются «близнецами», если модуль их
разности равен 2 (например, 41 и 43 — «близнецы»). Написать программу,
находящую все числа-близнецы, не превышающие $200$. (Определить
функцию для распознавания простых чисел.)

\task Написать программу, находящую периметр треугольника по
координатам его вершин. (Определить функцию для расчёта длины отрезка
по координатам его концов.)

\task Три прямые заданы в виде уравнений вида $ax+by+c=0$. Написать
программу, определяющую, сколько пар прямых имеет общие
точки. (Определить функцию, проверяющую, есть ли общие точки у двух
прямых.)

\task Три вектора на плоскости заданы своими координатами. Написать
программу, находящую пару векторов, образующих наименьший
угол. (Определить функцию, вычисляющую угол между векторами.)

\task Написать программу, находящую все трёхзначные числа, у которых
ровно $k$ делителей. (Определить функцию, находящую количество
делителей числа.)

\task Написать программу, находящую корни системы из двух линейных
уравненений с двумя неизвестными по заданным
коэффициентам. (Определить функцию, вычисляющую определитель матрицы
$2\times 2$.)

\subsection{Применение нескольких функций}

\task

\task

\task

\task

\task

\task

\task

\task

\task

\task

\subsection{Рекурсивные функции}

\task Написать рекурсивную функцию для вычисления количества цифр
десятичной записи натурального числа. Написать программу, использующую
эту функцию.

\task Квадратный корень произвольного действительного числа $a$ можно
вычислить при помощи итерационного метода Герона. Начальное
приближение $x_0 = 1,$ Каждое следующее вычисляется по формуле
\[
x_k = \frac12 \left( x_{k-1} + \frac{a}{x_{k-1}} \right).
\]
Итерации повторяются, пока $\left| x_k^2 - a \right| \geqslant
\varepsilon,$ где $\varepsilon > 0$ — некоторое малое число.  Написать
рекурсивную функцию, находящую приближённое значение $\sqrt{a}$ для
заданных $a$ и $\varepsilon.$ Написать программу, использующую эту
функцию.

\task Написать рекурсивную функцию для нахождения наибольшего общего
делителя двух чисел при помощи алгоритма Евклида. Написать программу,
использующую эту функцию.

\task Линейный конгруэнтный метод генерации псевдослучайных чисел
заключается в следующем. Выбирается произвольное число $x_0$, а каждое
следующее вычисляется по формуле
\[
x_k = (ax_{k-1} + b) \mod m,
\]
где коэффициенты выбраны особым образом. Например,
\[
a = 1664525,
b = 1013904223,
m = 2^{32}.
\]
Написать рекурсивную функцию, выводящую на экран $n$ псевдослучайных
чисел для заданного $x_0$.

\task Написать рекурсивную функцию для вычисления функции Аккермана
\[
A(m,n)=
\begin{cases}
  n+1,               &\textup{если }m=0;\\
  A(m-1, 1),         &\textup{если }m>0, n=0;\\
  A(m-1, A(m, n-1)), &\textup{если }m>0, n>0.
\end{cases}
\]
Написать программу, использующую эту функцию.

\task Последовательность Нарайаны определяется следующим образом:
\begin{align*}
  a_0 &= 1,\\
  a_1 &= 1,\\
  a_2 &= 1,\\
  a_k &= a_{k-1} + a_{k-3}.
\end{align*}
Написать функцию, находящую $k$-й член последовательности. Написать
программу, использующую эту функцию.

\task Написать рекурсивную функцию для нахождения максимального из $n$
чисел. Написать программу, использующую эту функцию.

\task Написать рекурсивную функцию находящую количество единиц в
двоичной записи заданного числа. Написать программу, использующую эту
функцию.

\task Написать рекурсивную функцию, которая находит максимальную цифру
десятичной записи указанного числа. Написать программу, использующую
эту функцию.

\task Для биномиальных коэффициентов верны следующие соотношения:
\begin{align*}
  C_n^k &= C_{n-1}^{k-1} + C_{n-1}^k,\\
  C_n^0 &= 1,\\
  C_n^n &= 1.
\end{align*}
Написать рекурсивную функцию для вычисления биномиальных
коэффициентов. Написать программу, использующую эту функцию.
