\section{Функции}

\subsection{Функции}

\task Даны координаты вершин двух треугольников. Написать
программу, проверяющую, лежит ли один треугольник внутри другого.
(Определить функцию для проверки, находится ли точка внутри
теугольника.)

\task Даны $n$ натуральных чисел. Написать программу, находящую их
общий делитель. (Определить функцию для расчёта наибольшего общего
делителя двух чисел.)

\task Даны три квадратных уравнения:
\begin{align*}
a x^2 + b x + c &= 0,\\
b x^2 + a x + c &= 0,\\
c x^2 + a x + b &= 0,
\end{align*}
где $a,b,c\neq 0.$
Написать программу, определяющую, сколько из них имеют действительные
корни. (Определить функцию, позволяющую распознавать наличие
вещественных корней в квадратном уравнении.)

\task Написать программу для вычисления биномиального коэффициента
\[
C_n^k = \frac{n!}{k!(n-k)!}
\]
для заданных $n$ и $k.$ (Определить функцию для вычисления факториала
числа.)

\task Два простых числа называются «близнецами», если модуль их
разности равен 2 (например, 41 и 43 — «близнецы»). Написать программу,
находящую все числа-близнецы, не превышающие $200.$ (Определить
функцию для распознавания простых чисел.)

\task Написать программу, находящую периметр треугольника по
координатам его вершин. (Определить функцию для расчёта длины отрезка
по координатам его концов.)

\task Три прямые заданы в виде уравнений вида $ax+by+c=0.$ Написать
программу, проверяющую, образуют ли они треугольник. (Определить
функцию, проверяющую, пересекаются ли две прямые ровно в одной
точке.)

\task Три вектора на плоскости заданы своими координатами. Написать
программу, находящую пару векторов, образующих наименьший
угол. (Определить функцию, вычисляющую угол между векторами.)

\task Написать программу, находящую все трёхзначные числа, у которых
ровно $k$ делителей. (Определить функцию, находящую количество
делителей числа.)

\task Написать программу, находящую корни системы из двух линейных
уравненений с двумя неизвестными по заданным
коэффициентам. (Определить функцию, вычисляющую определитель матрицы
$2\times 2.$)


\subsection{Функции высшего порядка}

\task Определить функцию, вычисляющую значение выражения
\[
\sum_{m=1}^n\bigg(
f(m)
\prod_{\substack{k=1,\\k\neq m}}^n
\frac{k}{k-m}
\bigg)
\]
для заданного натурального числа $n$ и функции $f.$ Написать
программу, использующую эту функцию.

Рассмотренное выражение — значение в нуле интерполяционного полинома
Лагранжа. Иными словами, программа будет по значениям
$f(1), f(2), \dots, f(n)$ приближённо находить значение $f(0).$

\task Уравнение вида $f(x)=x$ при $|f'(x)|<1$ может быть решено
методом простых итераций. В этом методе рассматриваются
последовательные приближения $x_k$ к корню уравнения. $x_0$ выбирается
произвольно, а каждое следующее приближение вычисляется по формуле
\[
x_{k+1} = f(x_k).
\]
Вычисления продолжаются, пока $|f(x_k) - x_k| \geqslant \varepsilon,$
где $\varepsilon$ — некоторое малое число.

Определить функцию, решающую уравнение по заданным $f$, $x_0$ и
$\varepsilon.$ Написать программу, использующую эту функцию.

\task Многие уравнения вида $f(x)=0$ можно решить методом хорд. Для
этого выбираются два числа $x_0$ и $x_1$ — концы отрезка, содержащего
корень. Затем выполняется последовательное приближение к корню:
\[
x_{k+1} = x_k - f(x_k)\frac{x_k - x_0}{f(x_k) - f(x_0)}.
\] 
Вычисления продолжаются, пока $|f(x_k)| \geqslant \varepsilon,$
где $\varepsilon$ — некоторое малое число.

Определить функцию, решающую уравнение по заданным $f$, $x_0$, $x_1$ и
$\varepsilon.$ Написать программу, использующую эту функцию.

\task Определить функцию, вычисляющую минимум средних
\[
\min_{\substack{1\leqslant m\leqslant n,\\ m\in\mathbb{N}}}
\left(
\frac{1}{m}\sum_{k=1}^{m} f(k)
\right)
\]
для заданного натурального числа $n$ и функции $f.$ Написать
программу, использующую эту функцию.

\task Определить функцию для приближённого вычисления значения
определённого интеграла методом левых прямоугольников по формуле
\[
\int\limits_a^b f(x) dx \approx
\frac{b-a}{n}\sum_{k=0}^{n-1} f \left(a + k\frac{b-a}{n}\right)
\]
для заданного натурального $n,$ функции $f$ и пределов интегрирования
$a$ и $b.$ Написать программу, использующую эту функцию.

\task Определить функцию для приближённого вычисления производной по
формуле
\[
f'(x) \approx \frac{-3f(x) + 4f(x+h) - f(x+2h)}{2h}
\]
для заданного малого числа $h,$ точки $x$ и функции $f.$ Написать
программу, использующую эту функцию.

\task Правой свёрткой некоторого списка
\[
a_0, a_1, a_2, \ldots, a_n
\]
с помощью функции $f(x, y)$ называется выражение
\[
f(a_0, f(a_1, f(\ldots f(a_{n-1}, a_n)) \ldots)). 
\]

Определить функцию, вычисляющую правую свёртку списка $1, 2, 3,
\ldots, n$ для заданного натурального числа $n > 2$ и функции
$f$. Написать программу, использующую эту функцию.

\task Определить функцию для приближённого вычисления значения
определённого интеграла методом трапеций по формуле
\[
\int\limits_a^b f(x) dx \approx
\frac{b-a}{n}\left(
\frac{f(a)+f(b)}2 + \sum_{k=1}^{n-1} f \left(a + k\frac{b-a}{n}\right)
\right)
\]
для заданного натурального $n,$ функции $f$ и пределов интегрирования
$a$ и $b.$ Написать программу, использующую эту функцию.

\task Левой свёрткой некоторого списка
\[
a_0, a_1, a_2, \ldots, a_n
\]
с помощью функции $f(x, y)$ называется выражение
\[
f(\ldots f(f(a_0, a_1), a_2) \ldots, a_n). 
\]

Определить функцию, вычисляющую левую свёртку списка $1, 2, 3, \ldots,
n$ для заданного натурального числа $n > 2$ и функции $f.$ Написать
программу, использующую эту функцию.

\task Определить функцию для приближённого вычисления значения
определённого интеграла методом центральных прямоугольников по формуле
\[
\int\limits_a^b f(x) dx \approx
\frac{b-a}{n}\sum_{k=0}^{n-1} f \left(
a + \left(k+\frac{1}{2}\right)\frac{b-a}{n}
\right)
\]
для заданного натурального $n,$ функции $f$ и пределов интегрирования
$a$ и $b.$ Написать программу, использующую эту функцию.


\subsection{Рекурсивные функции}

\task Определить рекурсивную функцию для вычисления количества цифр
десятичной записи натурального числа. Написать программу, использующую
эту функцию.

\task Квадратный корень произвольного действительного числа $a$ можно
вычислить при помощи итерационного метода Герона. Начальное
приближение $x_0 = 1,$ Каждое следующее вычисляется по формуле
\[
x_k = \frac12 \left( x_{k-1} + \frac{a}{x_{k-1}} \right).
\]
Итерации повторяются, пока $\left| x_k^2 - a \right| \geqslant
\varepsilon,$ где $\varepsilon > 0$ — некоторое малое число.

Определить рекурсивную функцию, находящую приближённое значение
$\sqrt{a}$ для заданных $a$ и $\varepsilon.$ Написать программу,
использующую эту функцию.

\task Определить рекурсивную функцию для нахождения наибольшего общего
делителя двух чисел при помощи алгоритма Евклида. Написать программу,
использующую эту функцию.

\task Линейный конгруэнтный метод генерации псевдослучайных чисел
заключается в следующем. Выбирается произвольное число $x_0,$ а каждое
следующее вычисляется по формуле
\[
x_k = (ax_{k-1} + b) \mod m,
\]
где коэффициенты выбраны особым образом. Например,
\[
a = 1664525,
b = 1013904223,
m = 2^{32}.
\]

Определить рекурсивную функцию, выводящую на экран $n$ псевдослучайных
чисел для заданного $x_0$.

\task Определить рекурсивную функцию для вычисления функции Аккермана
\[
A(m,n)=
\begin{cases}
  n+1,               &\textup{если }m=0;\\
  A(m-1, 1),         &\textup{если }m>0, n=0;\\
  A(m-1, A(m, n-1)), &\textup{если }m>0, n>0.
\end{cases}
\]
Написать программу, использующую эту функцию.

\task Последовательность Нарайаны определяется следующим образом:
\begin{align*}
  a_0 &= 1,\\
  a_1 &= 1,\\
  a_2 &= 1,\\
  a_k &= a_{k-1} + a_{k-3}.
\end{align*}

Определить функцию, находящую $k$-й член последовательности. Написать
программу, использующую эту функцию.

\task Определить рекурсивную функцию для нахождения максимального из
$n$ чисел. Написать программу, использующую эту функцию.

\task Определить рекурсивную функцию находящую количество единиц в
двоичной записи заданного числа. Написать программу, использующую эту
функцию.

\task Определить рекурсивную функцию, которая находит максимальную
цифру десятичной записи указанного числа. Написать программу,
использующую эту функцию.

\task Для биномиальных коэффициентов верны следующие соотношения:
\begin{align*}
  C_n^k &= C_{n-1}^{k-1} + C_{n-1}^k,\\
  C_n^0 &= 1,\\
  C_n^n &= 1.
\end{align*}

Определить рекурсивную функцию для вычисления биномиальных
коэффициентов. Написать программу, использующую эту функцию.
