\section{Параллельное программирование}

\subsection{Потоки и параллельные задачи}

\task Дана функция $f$ и массив действительных чисел. Написать
программу, находящую среднее значение функции от элементов
массива.

Вычисления распараллелить на несколько потоков. Сравнить время работы
программы при параллельных и последовательных вычислениях.

\task Написать программу, находящую сумму делителей числа, введённого
с клавиатуры.


Вычисления распараллелить на несколько потоков. Сравнить время работы
программы при параллельных и последовательных вычислениях.

\task Дан массив целых чисел. Написать программу, подсчитывающую
количество элементов, кратных числу, введённому с клавиатуры.

Вычисления распараллелить на несколько потоков. Сравнить время работы
программы при параллельных и последовательных вычислениях.

\task Написать программу, находящую приближённое значение
определённого интеграла функции $f$ по формуле Симпсона
\[
\int_a^b f(x)\,dx \approx \frac{h}3 \left(
  f(x_0) + 2\sum_{k=1}^{n-1} f(x_{2k}) + 4\sum_{k=1}^n f(x_{2k-1}) + f(x_{2n})
\right),
\]
где $h = \frac{b-a}{2n}, x_k=a+kh.$ 

Чем больше значение $n,$ тем выше точность результата.

Вычисления распараллелить на несколько потоков. Сравнить время работы
программы при параллельных и последовательных вычислениях.

\task Написать программу для транспонирования двумерного массива.

Вычисления распараллелить на несколько потоков. Сравнить время работы
программы при параллельных и последовательных вычислениях.

\task Написать программу для приближённого вычисления числа $\pi$
методом Монте-Карло (это группа методов для приближённого решения
математических задач с использованием случайных чисел).

В частности, для вычисления числа $\pi$ генерируются $N_0$ пар
равномерно распределённых случайных действительных чисел
$(x, y),$ где $x, y \in [-1, 1].$ Подсчитываются те пары, которые попадают в
единичный круг. То есть, те, для которых $x^2+y^2 \leqslant 1.$ Если
таких пар $N,$ то число $\pi$ можно оценить как
\[
\pi \approx \frac{4N}{N_0}.
\]

Вычисления распараллелить на несколько потоков. Сравнить время работы
программы при параллельных и последовательных вычислениях.

\task В компьютерной графике изображения часто представляются в виде
матрицы со значениями яркости отдельных пикселей. Один из методов
повышения чёткости изображения заключается в применении ко всем
элементам матрицы (кроме крайних) следующего преобразования:
\[
a'_{i,j} = 5a_{i,j} - \left(
  a_{i-1, j} + a_{i+1, j} + a_{i, j-1} + a_{i, j+1}
\right).
\]

Написать программу, применяющую это преобразование к заданной
двумерной матрице.

Вычисления распараллелить на несколько потоков. Сравнить время работы
программы при параллельных и последовательных вычислениях.

\task Для проверки, является ли число простым, можно воспользоваться
тестом Ферма. Если для нескольких произвольных чисел $a_i$, не
делящихся на $n$, верно, что
\[
a_i^{n-1} \equiv 1 \pmod n,
\]
то с определённой вероятностью число $n$ является простым. Чем больше
количество $a_i,$ тем больше шансов, что $n$ — простое число. Если же
хотя бы для одного $a$ окажется, что $a^{n-1}\not\equiv1 \pmod n,$ то
$n$ — составное число.

Написать функцию возведения целого числа в произвольную степень по
заданному модулю. Написать программу, выполняющую тест Ферма.

Проверки распараллелить на несколько потоков. Сравнить время работы
программы при параллельных и последовательных вычислениях.

\task Написать программу для параллельной сортировки коллекции. Для
этого коллекция делится на две подколлекции, каждая из которых
сортируется независимо и параллельно. Затем отсортированные подколлекции
объединяются в одну отсортированную.

Объединение (слияние) подколлекций можно выполнить следующим
образом. Элементы в начале каждой из них сравниваются, и выбирается
меньший элемент. Он извлекается и добавляется в конец
результата. Затем действия повторяются.

Сравнить время сортировки массива при параллельных и последовательных
вычислениях.

\task Написать программу, находящую скалярное произведение двух
векторов размерности $k$, представленных массивами.

Вычисления распараллелить на несколько потоков. Сравнить время работы
программы при параллельных и последовательных вычислениях.


\subsection{Задачи и асинхронное программирование}

\task

\task

\task

\task

\task

\task

\task

\task

\task

\task

